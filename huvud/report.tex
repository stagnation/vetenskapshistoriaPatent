\documentclass[12pt,a4paper]{article}
\usepackage[swedish]{babel}

%\usepackage{pdfpages}
\usepackage{NHQM}

\begin{document}

\title{Patentsystemets historia}
\author{Hampus Bengtsson \and Alexander Mihajlovic \and Nils Wireklint}


\maketitle
\selectlanguage{swedish}
\begin{abstract}
I rapporten redogörs för patentsystemets utveckling med avstamp i medeltidens England fram till det moderna patentsystemet. Rapporten redogör för vilka som styrt utvecklingen och vilka som har gynnats av systemets olika former och jämför dessa. Därutöver behandlas hur patent har används som ett styrmedel över marknaden historiskt och nu ett styrmedel av marknaden med handel av patent och lobbying för kapitalistisk patentlagstiftning. Slutligen reflkekter rapporten över patentsystemets allmännytta dels i att sprida kunskap och dels tillåta vanliga människor att ta patent och slå sig in på marknaden. Rapporten finner att patent är lämpligt för att sprida kunskap till allmänheten men att det fortfarande främst är företag som kan uttnytja patent fullt ut som ett strategiskt marknadsinstrument.
\end{abstract}
  
  
\numberwithin{equation}{section}
\numberwithin{figure}{section}

\listoftodos

%måste ha kina
%måste ha IP - ove granstrand.


\setcounter{page}{1}
\pagenumbering{roman}

\newpage
\tableofcontents

\newpage

%länder utan patent system?

\pagenumbering{arabic}

\subfile{../delar/inledning.tex}

\subfile{../delar/europa.tex}

\subfile{../delar/england.tex}

\subfile{../delar/usa.tex}

\subfile{../delar/nutid-usa.tex}

\subfile{../delar/gatt.tex}

\subfile{../delar/eu.tex}

\subfile{../delar/kina.tex}

\subfile{../delar/ip.tex}

\subfile{../delar/sammanfattning.tex}

\subfile{../delar/diskussion.tex}

\newpage
\bibliographystyle{ieeetr}
\bibliography{../huvud/patent.bib}


\end{document}