\section{sammanfattning}
\label{sec:samm}


\subsubsection{Medeltidens England}

Det medeltida privilegiesystemet för patent användes av kungar för att premiera landet över andra länder. Det åstadkomms genom att kungen kunde ge ut olika privilegier till de affärsmän som kunde realisera en gynsam marknadsnisch inom landet. En patentsökare var tydlig med att förklara nyttan av pattentet för allmänheten och också kungen, som såg till sitt eget intresse likväl folkets, men behövde ejspecificera upptäckten. Framförallt premierades import av utländsk teknologi och uppstart av inhemsk produktion och många patent hade krav på en vissmängd produktion eller att utländska handelsmän måste ta in engelska lärlingar för att lära styrka den inhemska marknaden. Detta under en protektionistiskt tid då alla sökte importera så lite som möjligt. Patent var oerhört dyra och krävde genomgick en ansträngande byråkrati.

\subsubsection{USA och England, 1850}

Vid mitten av 1800-talet har england och usa infört ett patentsystem som bygger på värderingen att en uppfinnnare har en naturligrätt till sin upptäckt. Ansöker om patent till ett särskiljt organ som objektivt undersöker om upptäckten är ny och beviljar sedan patentskydd i ett fixt antal år. Det har instiftats domstolar och juridiska medel för att lösa dispyt om patentintrång och för att underlääta detta krävs en patentspecifikation som beskriver upptäckten. Specifikationen tjänar också till att sprida kunskap till allmänheten och skulle bli väldigt viktig del av patentsystemet. England använde ett first to file system medan usa premierade first to invent. Patent var fortfarande relativt dyra men skulle snart bli billigare.

%\begin{itemize}
%	\item Patent beviljas subjektivt för främst uppfinningar och till viss grad metoder.
%	\item Patentansökan beviljas i stor utsträckning, efter att alla steg har tagits inom byråkratin
%	\item Patenttagaren måste lämna in en specifikation -inte alltid transparent.
%	\item Patentet ges fortfarande ut i fom av ett privilegie med olika utsträckning.
%	\item Patent undersökes grundligt vid tvist i domstol, först då kontrolleras patenterbarhet.
%	\item Case law och skriva lagen skiljer kraftigt.
%	\item Patentmål osäkra då dommare har fördommar och patent ej klart definierat.
%	\item Patent har varit del av en större dragkamp mellan kungen och regeringen om kungens privilegier. 
%	\item Saknades konkret patentprioritet.
%	\item Patent var oerhört dyra.
%\end{itemize}

%*eu



%\begin{itemize}
%	%\item Patent skall vara tillgängliga för uppfinningar (produket eller process) inom alla områden så länge ett "inventive step" har tagits och uppfinningen är applicerbar.
%	%\item Patentägaren har rätt att stoppa tredje part från att producera, sälja, importera skyddad produkt eller motsvarande för en process.
%	%\item Patenttagaren måste publicera uppfinngen tydligt.
%	\item Patentet varar i 20 år.
%	\item Växter och djur är ej patenterbara.
%	\item Datorprogram är ej patenterbara.
%\end{itemize}	


%*usa 1900-mitt
%
%\begin{itemize}
%	\item Patent ges för uppfinningar (produket eller process) efter objektiv revision i Patent Office.
%	\item Patent betraktas som uppfinnarens rättighet till sin idé.
%	\item Patenttagaren måste publicera uppfinngen tydligt.
%	\item Patentet kyddas i 14 år med förlängning till 21 år.
%	\item Patent ges ut till amerikaner och utlänningar.
%	\item Patntmål lyftes i regionala domstolar med olika perspektiv på patenträtt.
%	\item Patent prioriterades som first to invent
%\end{itemize}
%
%
%
%*trips

\subsection{Patenträtt i internationella handelsavtal, Intelectual Propety Rights}

I slutet av 1900-talet infördes flera internationella handelsavtal och dessa introducerade en
internationell syn på patent och \emph{intelectual property}. Patent skall ges till uppfinnare som gör en
ny teknologsik upptäckt. I utbyte mot att uppfinnare tydligt redogör för upptäckten i en
patentspecifikation får patenttagaren ensamrätt till upptäckten. %IPR ...

Kina har nyligen infört ett patentsystem i enlighet med internaitonella handelsavtal och patentansökningarna har senase årtiondet ökat dramatiskt. En stor del av dessa är patent från internationella företag som redan anökt om patentet i ett annat land.

%\begin{itemize}
%	\item Patent skall vara tillgängliga för uppfinningar (produket eller process) inom alla områden så länge ett "inventive step" har tagits och uppfinningen är applicerbar.
%	\item Patentägaren har rätt att stoppa tredje part från att producera, sälja, importera skyddad produkt eller motsvarande för en process.
%	\item Patenttagaren måste publicera uppfinngen tydligt.
%	\item Patentet ska vara skyddat i 20 år.
%	\item Patent skall ges ut på samma grund för medborgare i alla länder anslutna till WTO.
%	\item Patent kan ges för läkemdel.
%	\item Otydligt om mjukvara är patenterbart.
%\end{itemize}	
%%*icke-uppenbart
%%*table -stat mon -vendig -gammla usa? 
%%*läsa i berne convention? :(
%
%ovanstående är en sammanfattning av WTOs hemsida\cite{wto}.
%
%*usa nu:
%
%\begin{itemize}
%    \item{Alla, oavsett nationalitet, kan söka patent}
%    \item{Endast ursprungliga uppfinnaren kan söka patent}
%    \item{Internationella regler efterföljs, såsom skydd i (minst) 20 år}
%    \item{Patentansökningar utreds och godkänns av USPTO
%    \item{Även tvistemål utreds av USPTO}
%    \item{Tills alldeles nyligen tillämpades \emph{first to invent}, numera en variant av \emph{first to fi}}
%    \item{Patent ges för \emph{nya idéer} -- ej att de omsätts till praktik}
%\end{itemize}