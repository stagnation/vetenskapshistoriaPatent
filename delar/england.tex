\section{England} % (fold)
\label{sec:england}

\subsection{abskaffande av privilegier - Statue of Monopolies} % (fold)
\label{sub:abskaffande_av_privilegier_statue_of_monopolies}

England gav liberalt ut privilegier till handelsmän och hantverkare för att de skulle immigrera men är mest känd för att ge ut monopolsrättigheter till icke-legitima uppfinnare. Folket reagerade stark mot dessa godtyckliga och monopol som endast främjade monarkens egenintresse och till slut år 1624 revs alla kungliga monopol och privilegier upp av regeringen. De införde Statue of Monopolies som gav nya uppfiningar en mindre särstälning än tidigare monopol med strävade ändå till att uppmuntra uppfinnande. De tog inspiration av Europas nya patent lagstifningar och annamade i stort venedigs system. Det var fortfarande ett öppet brev från monarken, letter patent.

...


Dock fortsatte monarkerna att utge den gammla sortens monopol och detta blev en del av en större dragkamp mellan monarken och parlamentet, som representerade folket. 

% subsection abskaffande_av_privilegier_statue_of_monopolies (end)

\subsection{Patentsystemets uppkomst i London} % (fold)
\label{sub:patentsystemets_uppkomst_i_london}


på tidigt 1700 infördes ett krav på att specifiera hur uppfiningen fungerar, skrevs ned först 1734. Detta var ett första steg mot dagens uppfattning att patent ges av folket i utbyte mot att uppfinnaren eller upptäcktaren delger vad han kommit på och ger ut kunskapen i byte mot att ensamt få dra ekonomisk vinst under en tid. Men folket under 16- och 1700-talet var i princip bara londonbor. Innan 1852 fanns nämligen inget enkilt organ som kunde bevilja patent utan upphovsmannen var tvungen att vända sig till den kungliga byråkratin för att ansöka om ett patent. Denna process var väldigt lång och omfattande och kunde förväntas ta ett halvår och kosta flera hundra pund, i en tid då en etablerad hantverkare tjänade en eller två pund i månaden. Ingen gedigen undersökning genomfördes utan nästan alla patent godkändes så länge patentansökan ej inskränkte monarkens egen intresse eller tvivelaktiga egna monopol. Även om Nederländerna redan hade patentspecifikaiton som praxis var det England som var först att skriva det i lag. 

åter till specifikationen. ... 
Patentspecifikationen skulle bli en av englands viktigaste bidrag till moderna patentsystemet. Dock var specifikationen sällan tydlig utan då den endast undersöktes vid tvist kunde den ofta lämnas väldigt vag för att inte avslöja hemlighter om uppfinningen - detta går emot tanken att patent ges utbytte mot kunskap. Detta kunde dock få komplikationer om någon olovligt använde ett patent vars specifikation var för vag riskerade patentägaren att få patentet upprivet av den anledning om vederbörande drog intränsmannen till domstol. 

Att skriva en patentspecifikation blev således en knivig uppgift. Skrev man för vagt kunde hela patentet ogillas i en tvist men skrev man för precist gav man dels iväg alla hemligheter och dels kunde patentet kringås med hjälp av triviala ändringar. Det är av den här anledningen som James Watt var osäker på om ahn skulle stämma för patent intrång när X Y Z, osäkerheten speglas också i brevutbyte mellan Å Ä Ö. En blivande patenttagare fick numera anlita en skicklig specifikationskrivare innan han kunde ansöka om patent vilket gjorde processen än mer omfattande.

År 1753 genomfördes en till stor förändring av patentsystemet, nu kunde man ej längre ta patentintrång till the Privy Court utan var begränsad till Common Court. Privy Court var ??? monarkens domstol då patent beviljades av monarken kunde patentintrång anses som content of the royal perogative. Men 1753 klassades det om som ett civilt ärende och man var tvungen att ta patentintrångsmål genom den långsamma och dyra Common Court. Ytterligare så var det stor osäkerthet i vad ett patent juridiskt innebar och var gränserna för intrång gick, då case law* ännu ej blivit etablerat. Många dommare var också skeptiska mot allt vad egenrätt och patent hette då det påminde om monarkernas monopol.

Detta ledde till en osäker tid i patenträtt då många patenttagare var ovilliga att föra intrångsmän till rätta då de var rädda att domaren skulle ogilla patenten helt och hållet. Det var inte ovanligt att domarna helt slängde ut patentet på grund av trivialiteter i formuleringa eller till och med felkopieringar från clerkar. ??? . år 1795 skall Chief Justice Eyre i målet Voulton and Watt vs Bull ha sagt "patent rights are nowhere 
that I can find accurately described in our books"

% subsection patentsystemets_uppkomst_i_london (end)

\subsection{Patentsystemets spridning i landet} % (fold)
\label{sub:patentsystemets_spridning_i_landet}

Först mot slutet av 1700-talet började patent beviljas till aktörer utanför London. Detta hade flera anledningar, tidningar och handelsjournaler började skriva om nyutgivna patent, enklare transport och ökad handel inom landet gjorde patent mer relevanta. En positiv återkoppling uppstod, ju fler som tog patent desto fler hörde talas om patent och insåg att de själva behöver ta patent innan någon annan gör det. En annan bidragande faktor var arbetskraftskiftet ifrån lantbruk till andra nyligen proto-industrialiserade brancher, en liknande befolkningsförflyttning till städer hade liknande effekt; att fler blev medvetna om patent. För jordbruket hade historiskt sett väldigt lite patent, även om jordbruket omfattade en tredjedel av arbetskraften år 1800 stod det bara för fyra procent av patenten [macleod 1988].

Patentägarna var nu spridda mellan London och industriirterna Birmingham (metal), Lancashire samt Yorkshire (textil) och Nottingham(hosiers?). Ett belysande exempel är textilbranchen som under gillet främst sysslat med stugvävning, arbetarna satt i sina stugor och vävde, då var det snudd på omöjligt att kontrollera om någon använde din uppfinning, John Kay kunde inte uppsöka alla som olovligen använde hans flying shuttle för att få royalties. Men i och med centralisering av produktion och framförallt industrialiseringen kunde patent intrång lättare behandlas det fanns även större rum att göra egna uppfinningar eller förbättringar, varför patent blev mer önskvärda. Då industrin blomstrade var också varje enskilt patent mer värt då mer och mer royalties kunde förmodas i takt med produktionstillväxt.

statistik på det? 

england utanför patent finns lite omn i http://www.ipo.gov.uk/ipresearch-pindustrial-201011.pdf
% subsection patentsystemets_spridning_i_landet (end)

i fallet Kliardet v Johnson år 1778 fast slog det att patent beviljas den som redogör för uppfinningen till skillnad fårn den som gjorde uppfinningen. Väldigt viktig del av patentets case law och kan relateras till dagens diskusion om first-to-find och first-to-file. relaterar också tillbaka till då privilegier gavs till den som införde en uppfinning i området, ej den som annanstäders uppfann tinget ifråga.


% section england (end)