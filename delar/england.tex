\section{England} 

\subsection{Kungliga privilegier och patent} 

England gav liberalt ut privilegier till handelsmän och hantverkare för att de skulle immigrera men är
mest känd för att ge ut monopolsrättigheter till icke-legitima uppfinnare. Folket reagerade stark mot
dessa godtyckliga och monopol som främst främjade monarkens egenintresse och till slut år 1624 infördes
Statue of Monopolies som rev upp tidigare royal grants, förutom uppfinnings eller innovations patent.
Dock saknades ett mer rigidt regelverk för vilka patent inom detta område som skulle tillåtas, ett fokus
lades på bra monopol som gynnar allmänheten. De patent som gavs ut var fortfarande ett öppet brev från
monarken, letter patent, enligt gammal model.

Historiskt har handelsmonopol för en viss vara 
varit en del av monarkens möjliga privilegier att ge ut. Dessa behandlades
juridiskt som resten av hans privilegier. Varför patentintrång
behandlades som brott mot kronans privilegie, \emph{contempt of the royal prerogative} och behandlades i the prviy court \cite{macleod}. Denna
föregångare till patent kan ej förknippas med uppfinnarens 'naturliga' rätt till sin uppfinning eller idé
utan relaterade främst till kronans intresse och favoritism \cite{bracha}

Det samtida ordet 'invention' syftade ej till uppfinning utan närmre 'franchise' eller ny marknad, det
viktiga är att man kan kommersialisera något, och hur det är uppfunnit är inte relevant för patentet\cite{bracha}.
Därför kunde patent ges till engelsmän som införde utländska uppfinningar till den engelska marknaden.
Ett annat exempel på något som endast är en marknadsnisch är att försäkra hästar \cite{davies}.


*strukturera om inlednande tre stycken, grunda termer tydligare

Patentansökan addreserades till monarken och lämnades till dennes tjänstemän för hantering enligt ett
system från the Clerks Act från 1535\cite{bracha}. I ansökan listades uppfinningens fördelar för allmänheten men också
för kronan, och monarken vägde fördelarna för sig och allmänheten mot eventuella nackdelar innan patent
beviljades och de privilegier som patentet förde med sig var också bestämde för det speciella fallet. Det
kan anses att patent i sin första form snarare var ett handelskontrakt mellan patentagaren och kronan
\cite{bracha}. Vissa patent behandlade till och med en avgift som skulle erläggas kronan, \emph{rent},
det argumenteras dock att denna ej var en substantiell inkomstkälla till kungen. \cite{macleod2}

*rent > avgift?

Denna process var väldigt lång och omfattande och kunde
förväntas ta ett halvår och (tidigt 1700) kosta flera hundra pund, i en tid då en etablerad hantverkare tjänade en eller två pund i månaden. \cite{macleod}
Men ingen gedigen undersökning genomfördes utan nästan alla patent godkändes så
länge patentansökan ej inskränkte monarkens egen intresse eller tvivelaktiga egna monopol. 
Dessa patent kom att förknippas med favoritism och i viss mån korruption. Hela proccessen var på en
case-by-case nivå och de beviljade monopolen varrierade starkt i omfattning\cite{bracha}.

För att relatera till privilegierna för främmande handelsmän som skulle flytta in till landet för att
gynna landet så involverade många av dessa privilegiepatent vissa krav för att patenttagarens marknadsnisch verkligen realiserades. Det fanns \emph{working clause} som förmedlade att patentet ej var
gilltigt om patentagaren kommersialiserat upppfinningen, till exempel kunde kungen kontrollera att
tillräckligt mpnga varor producerats på utsatt tid. Fanns även \emph{apprentice claluse} som nästan
uteslutande berörde patent till utländska patentagare, de erfodrades att ta in engelska lärlingar och
lära ut sitt hantverk. Så med dessa föreskrifter kunde kungen använda monopol som ett verktyg för att styra marknaden och skynda på utveckling.\cite{bracha}

*term: privilegiepatent

Vi kan sammanfatta detta privilegiesystem med följande lista.
\begin{itemize}
	\item Patent ges av Kungen som ett subjektivt privilegie som han anser lämpligt.
	\item Patent ges för att kommersialisera en idé eller uppfinning, entreprenörskap.
	\item Patenttagaren måste förklara allmänhetens och kungens vinning tydligt.
	\item Patent gavs som ett ekonomiskt styrmedel.
	\item Patentet får individuelt anpassade privilegier, ibland monopol.
	\item Patent ges ofta för att locka till sig utläningar.
	\item Patent kan ges med produktions- och lärlingskrav.
\end{itemize}


\subsection{Statue of Monopolies} 
\label{sub:statue_of_monopolies}

\emph{Statue of Monopolies} kan anses vara ett första viktig steg från privilegier från kungen till ett
regelrätt patentsystem, cite. Men Bracha\cite{bracha} menar att det snarare var ett slag mot kungens abusiv grants
och endast var menat att begärnsa vilka monopol som beviljas, ej styra utvecklingen mot ett riktigt
system. Utan Macleod \cite{macleod2} säger att det var "a curious side effect a quirk of history" och endast
senare blev undantaget för uppfinningsmonopol (franchise) den viktigaste delen. Bracha ger en riklig redogörelse för den politiska dragkamp mellan regeringen och kronan som ledde fram till Staute of Monopolies. Detta bråk belyses i samtida historieskrivning; dommaren Lord Coke lyfter Statue of Monopolies som ett mått mot monopol o privilegier i stort, i Coke's Institues \cite{coke} enligt Bracha\cite{bracha}.

Baracha kommenterar att "first and true inventor" i sektion VI i Statue of Monopolies syftar till den som kommersialiserar ej upppfinner.
*omfattande breakdown av stat mon. tabel

Vidare säger Hulme att en uppfinning kunde anses vara ny även om det fanns tidigare skriven information om uppfinningen, om ingen redan lanserat uppfinningen kommersiellt \cite{hulme}.

En annan del av Statue of Monopolies förbjuder monopol för enbart förbättringar av befintlig teknik. Vilket med dagens perspektiv tycks konstigt, men Bracha menar att under en tid då patent gavs för marknadsnichar snarare än uppfinningar var det en rimligare begränsning.
Coke höll med om att en lite förbättring av existerande teknologi ej motiverade ett nytt monopol då det kunde skada existerande marknad och samhälle\cite{bracha}

Under sin regeringstidbörjade Charles 1 ge ut privilegiepatent som kringgick trivialiteter och loopholes i \emph{Staute of
Monopolies} ökade kritiken och slutet av privilegie patentet nalkades. Fox \cite{Fox} belyser hur dessa övertramp accelererade
konflikten mellan kungen och parlamentet som senare ledde till inbördes krig[???]. När dammet lagt sig
var alla monopol nedsmutsade och av mindre vikt. Kunde inte längre användas som ett verktyg för att styra
marknaden varför patent i allmänhet gavs mindre uppmärksamhet. Detta lyser igenom i Attorney General Yorkes uttalande år 1730: "it appears to me that patents of this kind for the sole use of manufacture newly brought into England and never before made here have formerly passed"\cite{macleod2}. 

Nu blev kungen mindre investerad i patentkontroll och nästan alla
patentansökningar godkändes \cite{bracha}. I det rådande politiska klimatet kunde han ej längre se till
egen vinning utan kungens egenintresse berörde nu bara patent som påverkade befinitliga inkomstskanaler
negativt\cite{macleod2}. Om kungen eller millitären berördes av en ansökan genomfördes dock en grundlig
undersökning. Likaså om en tvist uppstod antingen för patentintrång eller om någon ansåg att ett beviljat
monopol var olagligt (stred mot Statue of Monopolies) så genomfördes en undersökning för the privy
council. Man kan anse att denna undersökning motiverade en avslappnad behandling av ansökan.

----patentens utveckling i praktik, case law, specifikation.----

1700-talet var ett århundrade som kännetecknas av stora skillnader mellan den skrivna lagen och hur juridik praktiserades i domstolar. Den skrivna lagen var fortfarande \emph{Staute of Monopolies} men case law ändrade kontinuerligt det juridiska landskapet. Varför stor osäkerhet hos marknadsaktörerna råde.

I början av 1700-talet började uppfinnare beskriva sin uppfinningen med i specifikation i ansökan, detta blev ett nedskrivet krav 1734\cite{macleod}. 
Detta var ett första steg mot dagens uppfattning att patent ges av folket i utbyte mot att uppfinnaren eller
upptäcktaren delger vad han kommit på och ger ut kunskapen i byte mot att ensamt få dra ekonomisk vinst
under en tid. Folket i fråga var under 16- och 1700-talet i princip bara londonbor. MacLeod poängterar dock att detta ej var anledning till specifikationens uppkomst: "it was certainly not for the purpose of disseminating invention by disclosure" \cite{macleod2}. Åsikterna om varför specifikation intoduceras går isär: Hulme och Davies menar att det var ett steg patentagarna tog för att vara säkrare i tvister medan macleod menar att det var på initiativ av patentberedarna för att enklare kunna undersöka patentansökningar. Oavsett härkomst kom specifikationer att spella en central roll i patentsystemets framtida utveckling.

Under 1700-talets andra hälft förskjöts patent ytterliggare från att ge ut rätt till marknadsaktörer till att mer beröra uppfinningar som vi känner dem idag och i slutet av 1800-talet hade patent för marknadsnischar nästan helt försvunnit \cite{bracha}. 
Vid den här tidpunkten hade dock inga större förändringar till regelverket skett sedan \emph{Statue of Monopolies}
utan alla förändringar var i hur regelverket tolkades och vad \emph{innovation} innebär samt domstolars praxis och case law. Arbetet mot ett rigidt regelverk för patent började så smått år 1753 när historikern(????) Hulme \cite{hulme} observerar en "a reconsideration, from a constitutional standpoint, of the Council’s jurisdiction" varför Privy Court avsade sig rätten att ta upp patentmål. Många källor lyfter detta som ett markant steg mot ett allmänt patentsystem som behlandas i vanlig domstol. Men Bracha observerar att det var en mer diffus övergång och pekar på att densamme Hulme redogör för ytterligare patent mål i Privy Court efter 1753 \cite{bracha}.

Detta var, även om inte instantan, en stor förändring av patentsystemet, nu kunde man ej längre lyfta patentintrång i the Privy Court utan var begränsad till Common Court. Tidigare hade patentintrång ansets vara \emph{contempt of the royal perogative}. Men 1753 klassades det om som ett civilt ärende och man var tvungen att ta patentintrångsmål genom den långsamma och dyra Common Court. Ett viktigt beslut i case law var målet Morris v Bramson år 1776 där domare Mansfield fastslog att det rådande förbudet mot patent som förbättrar tidigare uppfinningar ej längre var relevant\cite{bracha}.

Mansfield dömde också i det omtalade fallet Johnson v Liardet år 1778 där han fastlog:
 -var det denna som ej är nedtecknad endast refererad?

"“for the condition of giving encouragement is this: that 
you must specify upon record your invention in such a 
way as shall teach an artist, when your term is out, to 
make it- and to make it as well as you by your directions: 
for then at the end of the term the public have the benefit 
of it. The inventor have the benefit during the term and 
the public have the benefit after.”" \cite{hulme}

Nu har patent helt skifftat från ett 'handelskontrakt' mellan kungen och en entreprenör till vår bild av ett kontrakt mellan allmänheten och en uppfinnare som delger sin upptäckt i utbyte mot att allmänheten låter denne ensam dra ekonomisk förtjänt av upptäcken. Specifikationen kom att bli en central del i detta synsätt: 

"for the end and meaning of the specification is, to teach the public, after the term for which the patent is granted, what the art is, ..." \cite{cases-davies} lösryckte från pp. 106 i cases oklart om citat eller författarens reflection.


“the specification is the price which 
the patentee is to pay for the monopoly.”-Justics Buller watt bull \cite{2 H. Bl. 463, 472, 126 Eng. Rep. 651, 656 (C.P. 1795)} English Reports

*specifikation publicerad direkt i ansökan (för alla) eller efter patenttiden?

Patentspecifikationen skulle bli ett av englands viktigaste bidrag till det
moderna patentsystemet. Även om
Nederländerna redan hade patentspecifikaiton som praxis var det England som var först att skriva det i
lag. \cite{macleod} Dock var specifikationen sällan tydlig utan då den endast undersöktes vid tvist
kunde den ofta lämnas väldigt vag för att inte avslöja hemlighter om uppfinningen - detta går emot tanken
att patent ges utbytte mot kunskap. Detta kunde dock få komplikationer om någon olovligt använde ett
patent vars specifikation var för vag, då riskerade patentägaren att få patentet upprivet av den anledningen om vederbörande drog intrångsmannen till domstol.

Att skriva en patentspecifikation blev således en knivig uppgift. Skrev man för vagt kunde hela patentet
ogillas i en tvist men skrev man för precist gav man dels iväg alla hemligheter och dels kunde patentet
kringås med hjälp av triviala ändringar\cite{macleod}. En
blivande patenttagare fick numera anlita en skicklig specifikationskrivare innan han kunde ansöka om
patent vilket gjorde processen än mer omfattande\cite{macleod}. Ytterligare så var det stor osäkerthet i vad ett patent juridiskt innebar och var gränserna för intrång gick, då case law ännu ej blivit etablerat.

Detta illustreras väl i fallet Boulton \& Watt v Bull år 1795, när Watt \& boulton stämmde Bull för patentintrång. Watt hade skrivit tidigare skrivit en vag specifikation om hans nya ångmotor, som endast beskrev en metod för att begränsa bränsle konsumption i en ångmotor, efter inrådan med patentadvokat\cite{bracha}. Fallet blev banbrytande då dommarna var tvungna att ta ställning till vare sig generella principer (till skillnad från konkret applikation) var patenterbara och om en metod kunde klassas som uppfinning\cite{bracha}. Detta skulle bli ett viktigt steg i en konkretiseringen av specifikationen och patenterbarhet. Den gängse bilden var att en innovation var en uppfinning och att man kunde patentera en sak \cite{bracha}. Varesig det gick att patentera en metod eller en process fick inget konkret svar. Vid tvistemål om vem som skulle få ett patent fanns ännu inget konkret prioriteringssystem varken first to find eller first to file\cite{bracha}.

*vad hände med watt?

*ej patentera naturlagar och abstrakta idéer 

*patents viktighet förtäljas i court of public opinion, bra uppfinning mycket pengardålig uppfinning inga pengar.

*patent som rättighet för en uppfinning.

*när kungen blev disintresserad av att undersöka patentansökan fförlorades mycket av privilegiestatusen -kungen brydde sig ju inte - rättighet


Kvarstod frågor om var gränsen för patenterbarhet gick och hur allting skulle tolkas\cite{bracha}. Många dommare var också skeptiska mot allt vad egenrätt och patent hette då det påminde om monarkernas monopol \cite{macleod}.
Det rådde en osäker tid i patenträtt då många patenttagare var ovilliga att föra intrångsmän till
rätta då de var rädda att domaren skulle ogilla patentet helt och hållet. Det var inte ovanligt att
domarna helt slängde ut patentet på grund av trivialiteter i formuleringa eller till och med
felkopieringar av tjänstemän\cite{macleod}. Vidare skulle Chief Justice Eyre i målet Boulton and Watt vs Bull år 1795 ha sagt "patent rights are nowhere that I can find accurately described in our books" \cite{macleod}. En senare observatör sade år 1829 "there being no existing basis of law, the dictum of the judge is 
one thing one day and another thing another‘" \cite{macleod}

tidiga 1800-talets patent system sammanfattas i en lista:

\begin{itemize}
	\item Patent beviljas subjektivt för främst uppfinningar och till viss grad metoder.
	\item Patentansökan beviljas i stor utsträckning, efter att alla steg har tagits inom byråkratin
	\item Patenttagaren måste lämna in en specifikation -inte alltid transparent.
	\item Patentet ges fortfarande ut i fom av ett privilegie med olika utsträckning.
	\item Patent undersökes grundligt vid tvist i domstol, först då kontrolleras patenterbarhet.
	\item Case law och skriva lagen skiljer kraftigt.
	\item Patentmål osäkra då dommare har fördommar och patent ej klart definierat.
	\item Patent har varit del av en större dragkamp mellan kungen och regeringen om kungens privilegier. 
	\item Saknades konkret patentprioritet, skulle bli first to file.
	\item Patent var oerhört dyra.
\end{itemize}

*även utveckling? 

Dock var folk inte nöjda med det rådande systemet och röster från hela spektrumet höjdes för patent reform. USA och Frankrike hade redan tagit stora steg för att garantera en uppfinnares rättighet till sin idé \cite{macleod}. Varför uppfinnare i England organiserade sig för att lobbya för en patent reform\cite{dutton}. Å andra sidan var det andra som höjde rösten för att fullständigt upplösa patensystemet, denna grupp involverade också framträdande ingenjörer \cite{macleod}.  
Som svar genomfördes år 1852 en stor reform av patentsystemet, då introducerades The Patent Law Amendment Act, 1852. Nu infördes ett samlat patent för hela storbritanien med en enhetlig ansökansprocces och kostnaden för ett patent delades upp i tre delar som betalades vid ansökan och under de första sju åren av ett patent. Och ett system för att anslå godkända patent för att enklare nå ut med kunskap till folket\cite{dutton}.

Men det fanns också folk som höjde rösterna mot införandet av det nya systemet. Sockerindustrin i London och Glasgow samt Manchester Chamber of Commerce var strikt emot en föreslagen sänkning av patentkostnaden då alla triviala ändringar således skulle patenteras och de menade att dessa snarare skulle hindra utveckling\cite{dutton}. Men motsätningar var endast en minioritet och reformen faststäldes i lag. I lagen fastslogs priset till 180 pund vilket var högre än tidigare pris men uppdelat\cite{dutton}. 

*kan inte relatera pengar till något men det är fortfarande väldigt dyrt

Mycket av kritiken emot tillika det gamla systemet och reformen från olika håll dog ner men det var erkännt från regeringens sida att ämnet skulle tas upp igen och att det inte var en slutgiltig lösning\cite{dutton}. Men systemet är tillräckligt likt det moderna systemet och vid slutet av industriella revolutione avslutar de flesta källor historieskrivningen vid den här tidpunkten.

*first to file, inget system innan 18-mitten


\subsection{Patentsystemets spridning i landet}
\label{sub:patentsystemets_spridning_i_landet}

vvvvvvvvvvv----macleod----
Först mot slutet av 1700-talet började patent beviljas till aktörer utanför London. Detta hade flera
anledningar, tidningar och handelsjournaler började skriva om nyutgivna patent, enklare transport och
ökad handel inom landet gjorde patent mer relevanta\cite{macleod}. Macleod observerar att en positiv återkoppling uppstod, ju fler som tog
patent desto fler hörde talas om patent och insåg att de själva behöver ta patent innan någon annan gjorde
det. En annan bidragande faktor var arbetskraftskiftet ifrån lantbruk till andra nyligen
proto-industrialiserade brancher, en liknande befolkningsförflyttning till städer hade liknande effekt;
att fler blev medvetna om patent\cite{macleod}. För jordbruket hade historiskt sett väldigt lite patent, även om
jordbruket omfattade en tredjedel av arbetskraften år 1800 stod det bara för fyra procent av patenten
\cite{macleod2}.

Patentägarna var nu spridda mellan London och industriorterna Birmingham (metal), Lancashire samt
Yorkshire (textil) och Nottingham (trikå). Ett belysande exempel är textilbranchen som under gillet
främst sysslat med stugvävning, arbetarna satt i sina stugor och vävde, då var det svårt att
kontrollera om någon använde en uppfinning, John Kay kunde exempelvis inte uppsöka alla som olovligen
använde hans \emph{flying shuttle} för att få royalties \cite{macleod}. Men i och med centraliseringen och industrialiseringen av produktion kunde patentintrång lättare behandlas. Det fanns även större rum att göra
egna uppfinningar eller förbättringar, varför patent blev mer önskvärda. Då industrin blomstrade var
också varje enskilt patent mer värt då mer och mer royalties kunde förmodas i takt med
produktionstillväxt.


*england utanför patent finns lite omn i http://www.ipo.gov.uk/ipresearch-pindustrial-201011.pdf <- macleod1


*glöm inte bort kollektiv uppfinning och rörelse för att avskaffa patent 
*james watt ångmaskin - stagnation inte utveckling


