\section{England} % (fold) \label{sec:england}

\subsection{abskaffande av privilegier - Statue of Monopolies} % (fold)
\label{sub:abskaffande_av_privilegier_statue_of_monopolies}

England gav liberalt ut privilegier till handelsmän och hantverkare för att de skulle immigrera men är
mest känd för att ge ut monopolsrättigheter till icke-legitima uppfinnare. Folket reagerade stark mot
dessa godtyckliga och monopol som främst främjade monarkens egenintresse och till slut år 1624 infördes
Statue of Monopolies som rev upp tidigare royal grants, förutom uppfinnings eller innovations patent.
Dock skapades en mer rigidt regelverk för vilka patent inom detta område som skulle tillåtas, ett fokus
lades på bra monopol som gynna allmänheten. De patent som gavs ut var fortfarande ett öppet brev från
monarken, letter patent, enligt gammal model.

Då handelsmonopol för en viss vara endast var en del av monarkens möjliga privilegier att ge ut, varför
de juridiskt behandlades som vilket annnat av hans privilegier som helst. Varför patentintrång
behandlades som brott mot kronans privilegies \emph{contempt...} och behandlades i the prviy court. Denna
föregångare till patent kan ej förknippas med uppfinnarens 'naturliga' rätt till sen uppfinning elelr idé
utan relaterade främst till kronans intresse och favoritism \cite{obracha}

Det samtida ordet 'invention' syftade ej till uppfinning utan närmre 'franchise' eller ny marknad, det
viktiga är att man kan kommersialisera något, och hur det är uppfunnit är inte relevant för patentet.
Därför kunde patent ges till engelsmän som införde utländska uppfinningar till den engelska marknaden.
Ett annat exempel på något som endast är en marknadsniche är att försäkra hästar. \cite{davies}

Patentansökan addreserades till monarken och lämnades till dennes tjänstemän för hantering enligt ett
system från the Clerks Act från 1535. I ansökan listades uppfinningens fördelar för allmänheten men också
för kronan, och monarken vägde fördelarna för sig och allmänheten mot eventuella nackdelar innan patent
beviljades och de privilegier som patentet förde med sig var också bestämde för det speciella fallet. Det
kan anses att patent i sin första form snarare var ett handelskontrakt mellan patentagaren och kronan
\cite{obracha}. Vissa patent behandlade till och med en avgift som skulle erläggas kronan, \emph{rent},
det argumenteras dock att denna ej var en substantiell inkomst kökka till kungen. \cite{macleod2}

Dessa patent kom att förknippas med favoritism och i viss mån korruption. hela proccessen var på en
case-by-case nivå och de beviljade monopolen varrierade starkt i omfattning.-> statue of mon.

För att relatera till privilegierna till främmande handelsmän som skulle flytta in till landet för att
gynna landet så involverade många patent vissa krav för att marknadsnichen som patenttaharen skulle ta på
sig att fylla verkligen realiserades. Det fanns \emph{working clause} som förmedlade att patentet ej var
gilltigt om patentagaren kommersialiserat upppfinningen, till exempel kunde kungen kontrollera att
tillräckligt mpnga varor producerats på utsatt tid. Fanns även \emph{apprentice claluse} som nästan
uteslutande berörde patent till utländska patentagare, de erfodrades att ta in engelska lärlingar och
lära ut sitt hantverk. Så med dessa föreskrifter kunde kungen använda monopol för att styra marknaden och
sktnda på utveckling.

term: privilegia patent

Statue of Monopolies kan anses vara ett första viktig steg från privilegier från kungen till ett
regelrätt patent system, cite. men Obracha menar att det snarare var ett slag mot kungens abusiv grants
och endast var menat att begärnsa vilka monopol som beviljas, ej styre utvecklingen mot ett riktigt
system. Macleod \cite{macleod2} säger att det var "a curious side effect a quirk of history" endast
senare blev undantaget för uppfinningsmonopol (franchise) den viktigaste delen, samtida historieskrivning
belyser att det är ett mått mot monopol o privilegier i stort, Coke's Institues \cite{coke}.

för en förtäljning av bråket mellan common law och kung som ledde upp till stat mon se obracha

section VI stat mon. "first and true inventor" syftar till den som kommersialiserar ej upppfinner.

Charles 1 började senare började ge ut patent som kringgick trivialiteter och loophikes i Staute of
Monopolies närmas slutet av privilegie patentet. \cite{Fox} belyser hur dessa övertramp accelererade
konflikten mellan kungen och parlamentet som senare ledde till inbördes krig. [???] När dammet lagt sig
ned var alla monopol nedsmutsade och av mindre vikt. Kunde inte längre användas som verktyg för att styra
marknade varför patent i allmänhet gavs mindre uppmärksamhet. I stället användes Parliamentary
legislation and tariffs. 116

Därmed (andra hälft 1700) förskjöts patent från att ge ut rätt till marknadsaktörer till att mer beröra
uppfinningar som vi känner dem idag. Nu blev kungen mindre investerad i patentkontroll och nästan alla
patentansökningar godkändes \cite{ochra}. I det rådande politiska klimatet kunde han ej längre se till
egen vinning utan kungens egenintresse berörde nu bara patent som påverkade befinitliga inkomstskanaler
negativt. \cite{macleod2} Om kungen eller millitären berördes av en ansökan genomfördes dock en grundlig
undersökning. Likaså om en tvist uppstod antingen för patentintrång eller om någon ansåg att ett beviljat
monopol var olagligt (stred mot Statue of Monopolies) så genomfördes en undersökning för the privy
council. MAn kan anse att denna fallback undersökning gjorde en inledande undersökning mindre viktig.

Vid den här tidpunkten hade inga större förändringar till regelverket skett sedan Statue of Monopolies
utan alla förändringar var i hur regelverket tolkades och vad \emph{innovation} innebär samt domstolls
praxis

vilken domstol när?

% subsection abskaffande_av_privilegier_statue_of_monopolies (end)

\subsection{Patentsystemets uppkomst i London} % (fold) \label{sub:patentsystemets_uppkomst_i_london}

på tidigt 1700 infördes ett krav på att specifiera hur uppfiningen fungerar, skrevs ned först 1734. Detta
var ett första steg mot dagens uppfattning att patent ges av folket i utbyte mot att uppfinnaren eller
upptäcktaren delger vad han kommit på och ger ut kunskapen i byte mot att ensamt få dra ekonomisk vinst
under en tid. Men folket under 16- och 1700-talet var i princip bara londonbor. Innan 1852 fanns nämligen
inget enkilt organ som kunde bevilja patent utan upphovsmannen var tvungen att vända sig till den
kungliga byråkratin för att ansöka om ett patent. Denna process var väldigt lång och omfattande och kunde
förväntas ta ett halvår och kosta flera hundra pund, i en tid då en etablerad hantverkare tjänade en
eller två pund i månaden. Ingen gedigen undersökning genomfördes utan nästan alla patent godkändes så
länge patentansökan ej inskränkte monarkens egen intresse eller tvivelaktiga egna monopol. Även om
Nederländerna redan hade patentspecifikaiton som praxis var det England som var först att skriva det i
lag.

åter till specifikationen. ... Patentspecifikationen skulle bli ett av englands viktigaste bidrag till
moderna patentsystemet. Dock var specifikationen sällan tydlig utan då den endast undersöktes vid tvist
kunde den ofta lämnas väldigt vag för att inte avslöja hemlighter om uppfinningen - detta går emot tanken
att patent ges utbytte mot kunskap. Detta kunde dock få komplikationer om någon olovligt använde ett
patent vars specifikation var för vag riskerade patentägaren att få patentet upprivet av den anledning om
vederbörande drog intränsmannen till domstol.

Att skriva en patentspecifikation blev således en knivig uppgift. Skrev man för vagt kunde hela patentet
ogillas i en tvist men skrev man för precist gav man dels iväg alla hemligheter och dels kunde patentet
kringås med hjälp av triviala ändringar. Det är av den här anledningen som James Watt var osäker på om
ahn skulle stämma för patent intrång när X Y Z, osäkerheten speglas också i brevutbyte mellan Å Ä Ö. En
blivande patenttagare fick numera anlita en skicklig specifikationskrivare innan han kunde ansöka om
patent vilket gjorde processen än mer omfattande.

År 1753 genomfördes en till stor förändring av patentsystemet, nu kunde man ej längre ta patentintrång
till the Privy Court utan var begränsad till Common Court. Privy Court var ??? monarkens domstol då
[vilken domstol när?] patent beviljades av monarken kunde patentintrång anses som content of the royal
perogative. Men 1753 klassades det om som ett civilt ärende och man var tvungen att ta patentintrångsmål
genom den långsamma och dyra Common Court. Ytterligare så var det stor osäkerthet i vad ett patent
juridiskt innebar och var gränserna för intrång gick, då case law* ännu ej blivit etablerat. Många
dommare var också skeptiska mot allt vad egenrätt och patent hette då det påminde om monarkernas monopol.

Detta ledde till en osäker tid i patenträtt då många patenttagare var ovilliga att föra intrångsmän till
rätta då de var rädda att domaren skulle ogilla patenten helt och hållet. Det var inte ovanligt att
domarna helt slängde ut patentet på grund av trivialiteter i formuleringa eller till och med
felkopieringar från clerkar. ??? . år 1795 skall Chief Justice Eyre i målet Voulton and Watt vs Bull ha
sagt "patent rights are nowhere that I can find accurately described in our books"

% subsection patentsystemets_uppkomst_i_london (end)

\subsection{Patentsystemets spridning i landet} % (fold) \label{sub:patentsystemets_spridning_i_landet}

Först mot slutet av 1700-talet började patent beviljas till aktörer utanför London. Detta hade flera
anledningar, tidningar och handelsjournaler började skriva om nyutgivna patent, enklare transport och
ökad handel inom landet gjorde patent mer relevanta. En positiv återkoppling uppstod, ju fler som tog
patent desto fler hörde talas om patent och insåg att de själva behöver ta patent innan någon annan gör
det. En annan bidragande faktor var arbetskraftskiftet ifrån lantbruk till andra nyligen
proto-industrialiserade brancher, en liknande befolkningsförflyttning till städer hade liknande effekt;
att fler blev medvetna om patent. För jordbruket hade historiskt sett väldigt lite patent, även om
jordbruket omfattade en tredjedel av arbetskraften år 1800 stod det bara för fyra procent av patenten
[macleod 1988].

Patentägarna var nu spridda mellan London och industriirterna Birmingham (metal), Lancashire samt
Yorkshire (textil) och Nottingham(hosiers?). Ett belysande exempel är textilbranchen som under gillet
främst sysslat med stugvävning, arbetarna satt i sina stugor och vävde, då var det snudd på omöjligt att
kontrollera om någon använde din uppfinning, John Kay kunde exempeliv inte uppsöka alla som olovligen
använde hans \emph{flying shuttle} för att få royalties. Men i och med centralisering av produktion och
framförallt industrialiseringen kunde patent intrång lättare behandlas det fanns även större rum att göra
egna uppfinningar eller förbättringar, varför patent blev mer önskvärda. Då industrin blomstrade var
också varje enskilt patent mer värt då mer och mer royalties kunde förmodas i takt med
produktionstillväxt.

statistik på det?

england utanför patent finns lite omn i http://www.ipo.gov.uk/ipresearch-pindustrial-201011.pdf %
subsection patentsystemets_spridning_i_landet (end)

i fallet Kliardet v Johnson år 1778 fast slog det att patent beviljas den som redogör för uppfinningen
till skillnad från den som gjorde uppfinningen. Väldigt viktig del av patentets case law och kan
relateras till dagens diskusion om first-to-find och first-to-file. relaterar också tillbaka till då
privilegier gavs till den som införde en uppfinning i området, ej den som annanstäders uppfann tinget
ifråga.

*glöm inte bort kollektiv uppfinning och rörelse för att avskaffa patent james watt ångmaskin -
stagnation inte utveckling

% section england (end)
