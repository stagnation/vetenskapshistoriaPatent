
\subsection{Internationella Handelsavtal}
Ett stort steg från separata patentsystem, av olika karaktär, i olika länder till ett eller två mer enade moderna patentsystemet som respekterar patent från annan härkomst togs 1947 när \emph{General Agreement on Tariffs and Trade}, GATT överenskoms. Detta var en handelsavtal mellan de flesta stora länder som för att främja rättvis handel mellan länderna \cite{gatt}. Avtalet reglerar de flesta vanliga handelsbarriärer som ett enskilt land kan använda för att skydda den inhemska marknaden, exempevis skyddstullar. En annan sorts handelsbarriär är upphovsrättsstöld och -intrång, det går inte att etablera en marknad i ett land som ej repsekterar ett företags patent. Detta gör att företag har svårare att ta steget från nationel marknad till den internationella spelplanen.

GATT avtalet uppdaterades kontinuerligt och den viktigaste uppdeteringen kom i Uruguay-rundan 1986-1994. Här instiftades \emph{Agreement on Trade Related Aspects of Intellectual Property Rights} TRIPS som stärker patent på en internationell spelplan. TRIPS föreskriver att man skall kunna patentera applicerbara uppfinningar och nya metoder inom alla tekniska områden. Patentägaren har ensamrätt att till att sälja och producera sin produkt. I gengäld skall patentet grundligt specifiseras så att kunskapen delges till allmänheten. Patentskyddet skall vara  i minst 20 år och uppfinnare från alla anslutna länder skall ha samma möjlighet att söka patent. Ovanstående är en sammanfattning av WTOs hemsida\cite{wto}.

TRIPS kom till efter ihärdig lobbying från främst USA men också EU och Japan\cite{drahos}. Läkemedelsföretag som Pfizer var kraftigt involverade i denna lobbying\cite{drahos}. Drahos menar att Pfizer vid den här tidpunkten hade mycket att vinna på starkare internationellt patent skydd då de precis investerat i läkemdelsproduktion i flera utvecklingsländer och såg att Indien skulle kunna konkurrera med billigare mediciner\cite{drahos}. Pfizer övertalade viktiga positioner i USA att skulle gynna landets ekonomi att stärka patentskydd och \emph{Intelectual Property} som en investeringsmodel och så småningom även Europa och Japan\cite{drahos}.
%mer mer mer

Dessutom skapades i Uruguayrundan \emph{World Trade Organization}. Denna organisation ersätter GATT i att reglera internationell handel och tillgodose ett forum för förhandlingar och att lösa tivster, men GATT är endast en av stöttepelarna i WTO, även TRIPS är representerad. Detta är viktigt då alla länder som vill vara delaktiga i WTO och i förlängningen internationell handel måste även införa ett patentsystem som följer alla TRIPS:s riktlinjer. Nästan alla länder är i dag medlemmar i WTO och forskning visar att Uruguay rundan, WTO i stort, har varit väldigt bra för välfärden\cite{harrison}. 

Då utvecklingsländer ofta saknat den rigida upphovsrättslagstiftning som sedan länge funnits i västvärlden har de ofta tagit inspiration från västvärldens system \cite{finger}, vilket ger USA ytterliggare internationellt inflyttande då andra länder delvis implementerar schablonkopior. Således finner sig många u-länder i en situation där de måste acceptera det (främst) amerikanska patentperspektivet för att få vara med i handelsorganisationerna, kan se som en form av utpressning.

%*detta leder till utpressningssituation för u-länder även om de fått lite dispens.
