\section{Moderna patentsystem -et}

\subsection{Internationella Handelsavtal}
Ett stort steg från individuella patentsystem, av olika karaktär, i olika länder till ett eller två mer enade moderna patentsystemet som respekterar patent från annan härkomst togs 1947 när \emph{General Agreement on Tariffs and Trade}, GATT överenskoms. Detta var en handelsavtal mellan de flesta stora länder som för att främja rättvis handel mellan länderna. \cite{gatt} Avtalet reglerar de flesta vanliga handelsbarriärer som ett enskilt land kan anvämda för att skydda den inhemska marknaden, exempevis skyddstullar. En annan handelsbarriär är upphovsrättsstöld och -intrång, 

GATT avtalet uppdaterades kontinuerligt och den viktigaste uppdeteringen kom i Uruguay-rundan 1986-1994. Här instiftades \emph{Agreement on Trade Related Aspects of Intellectual Property Rights} TRIPS som stärker patent på en internationell spelplan. Detta avtal föreskriver att deltagarländer tillhandahar intern copyright upphovsrätt där bland annat patent innefattas. TRIPS föreskriver att patentsystem bland annat ska ha:

\begin{itemize}
	\item Patent skall vara tillgängliga för uppfinningar (produket eller process) inom alla områden så länge ett "inventive step" har tagits och uppfinningen är applicerbar.
	\item Patentägaren har rätt att stoppa tredje part från att producera, sälja, importera skyddad produkt eller motsvarande för en process.
	\item Patenttagaren måste publicera uppfinngen tydligt.
	\item Patentet ska vara skyddat i 20 år.
	\item Patent skall ges ut på samma grund för medborgare i alla länder anslutna till WTO.
	\item Patent kan ges för läkemdel.
	\item Otydligt om mjukvara är patenterbart.
\end{itemize}	
*icke-uppenbart
*table -stat mon -vendig -gammla usa? 
*läsa i berne convention? :(

ovanstående är en sammanfattning av WTOs hemsida\cite{wto}. TRIPS kom till efter ihärdig lobbying från främst USA\cite{drahos} men också EU och Japan. Läkemedelsföretag som Pfizer var kraftigt involverade i denna lobbying\cite{drahos}.


Dessutom skapades i Uruguayrundan \emph{World Trade Organization}. Denna organisation ersätter GATT i att reglera internationell handel och tillgodose ett forum för förhandlingar och att lösa tivster, men GATT är endast en av stöttepelarna i WTO även TRIPS är representerad. Detta är viktigt då alla länder som vill vara delaktiga i WTO och i förlängningen internationell handel måste även införa ett patent system som följer alla TRIPS:s riktlinjer. Nästan alla länder är i dag medlemmar i WTO och forskning visar att Uruguay rundan, WTO i stort, har varit väldigt bra för välfärden\cite{harrison}. 

Då utvecklingsländer ofta saknat den rigida upphovsrättslagstifning som sedan länge funnits i västvärlden har de ofta tagit inspiration från västvärldens system\cite{finger}, vilket ger USA ytterliggare internationellt inflyttande då andra länder delvis implementerar schablonkopior.

*detta leder till utpressningssituation för u-länder även om de fått lite dispens.