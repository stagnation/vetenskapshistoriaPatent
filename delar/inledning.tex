\begin{abstract}
	hej hopp
\end{abstract}	
	
	
\section{inledning}
Introduktion av det problemområde eller den företeelse som skall studeras. 
Definition av patent. Trivia om Sybaris.

Arbetets syfte är att utforska rötterna till dagens patentsystem, och att ta
reda på vad patentsystemets ursprungliga 

reda på hur väl det moderna patentsystemet fyller sitt syfte. När började man 
utfärda patent? Varför utfärdade man patent? Vilka alternativ fanns innan 
patent? 

Syfte med klart definierade frågeställningar

Kort metoddiskussion om källmaterialet

Våra främsta källor för patentsystemets historia är MacLeod \cite{macleod}, 
Bracha \cite{bracha} och Nard \cite{nard}. Chrisitne MacLeod sammanfattar 
brittiska patentsystemets utveckling mellan 1600 och 1900 i rapporten \emph{
Patents and Industrialisation - An Historical Overview of the British Case, 
1624-1907}, vi fann denna på brittiska Intellectual Property Offices hemsida 
och drar slutsatsen att detta är en rapport skriven för IPO. MacLeod är 
historie professor vid University of Bristol i England varför vi finner 
rapporten trovärdig. Oren Bracha ger en detaljrik och klarsynt bild av patent 
systemets utveckling från monarkiska privilegierna i England i hans 
doktorsavhandling \emph{Owning Ides: A History of Angle-American Intellectual 
Property} Bracha är nu juridik proffesor vid University of Texas School of Law
. Avhandlingen citerar ofta ett av annat av MacLeods verk så dessa två källor 
tar del av samma historieskrivning, vilket inte är så konstigt då få 
diskrepanser har upptäckts mellan olika källor. Craig Allen NArd har skrivit 
boken \emph{The law of patents} som främst behandar juridiken bakom 
patentsystemet idag men också viger ett kapitell åt patentsystemets uppkomst, 
Nard är juridik professor vid Case Western Reserve. Dessa källor anses 
trovärdiga då samtliga författare är erkända professorer inom sina fält och 
texterna i sig uppvisar hög akademisk standard. I alla de fall som dessa 
huvudkällor refererar vidare till andra verk för specifika detaljer har vi 
sökt att läsa orginalkällan men i de fall dessa ej varit tillgängliga har vi 
litat på andra hands källor.
  

Definition av patent.

Termen patent kommer från latinets \emph{litterae patentes} som betyder öppet 
brev. Kopplingen till dagens patent är att medeltida monarker gav ut
'privilegier', land, titlar, friheter och informerade om detta genom ett öppet 
brev, \emph{litterae patentes}, där det kungliga sigillet placerades på ett
sådant sätt att brevet skulle kunna läsas utan att bryta sigillet (jmf: 
vanligt, stängt brev). Breven var också uttryckligen addreserade till 
vemhelst som läser det, alltså alla. Dessa privilegier kunde innefatta ensamrätt
att sälja en viss vara. Ordet patent har levt kvar genom historien till 
att nu innebära kommersiell ensamrätt till främst uppfinningar men även 
design etc. \cite{blackstone vad hänvisas?}

när vi talar om patent i historisk mening är det viktigt att poängtera att det 
snarare är \emph{litterae patentes} med koppling till de tidigare 
preivilegierna än moderna patatent. Första delen i rapporten beskriver vad vi 
idag skulle kalla patents härkomst ifrån de sentida \emph{litterae patentes}.

diskusion om källor, viktning, trovärdighet.

