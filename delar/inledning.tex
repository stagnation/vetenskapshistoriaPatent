
\begin{abstract}
	hej hopp
\end{abstract}	
	
	
\section{Inledning}

Patent har antagit en central roll i den globala marknaden.
Det ansöks om fler patent än någonsin, och i en så hög takt att patentutfärdarna
får svårt att hinna med \cite{uspto_number}.
Patent har blivit ett viktigt strategiskt verktyg för företag att skydda sina
intressen, stärka sin förhandlingsposition, generera inkomster genom
licensering, och att hindra konkurrenter från att äntra och utvecklas på
marknaden. 
Den senaste tiden har flera stora patentstrider mellan teknikföretag
uppmärksammats i media, och detta har gett liv åt debatten om vilken nytta
patent tillför samhället.
Att patent har stor betydelse i den globala ekonomin ses tydligt genom att
de utvecklade länderna drev på införandet av starkare patentskydd i
internationella förhandlingar \cite{}. 

Ett patent är en ensamrätt att utnyttja en ny uppfinning. Det innebär att ingen
annan får utnyttja uppfinningen genom att tillverka, sälja eller importera 
uppfinningen utan patentinnehavarens tillstånd. Patent är en av flera sorters
immaterialrätt, men vi har valt att begränsa vår studie till patent enligt
definitionen ovan, och undviker att behandla närliggande koncept som upphovsrätt
etc.

Arbetets syfte är att utreda det historiska ursprunget till patentsystemet och
granska det kritiskt i ljus av den aktuella debatten kring patent.
Vi vill också utreda hurvida maximet att ett patent är ett kontrakt mellan en
uppfinnare och allmänheten där allmänheten tillåter att uppfinnaren har ensam
ekonomisk vinning av uppfinningen i utbyte mot att han delger kunskap till
allmänheten, public domain, har historisk förankring.

Alltså vill vi besvara frågan om patent historiskt uppkommit för att gynna
allmänheten eller av annan anledning. Våra övriga frågeställningar inkluderar:
När började patent att utfärdas? Vilka alternativ fanns innan patent?
Vilka har haft makt över, och vilka har gynnats av patentsystemet?


\subsection{Metod och källor}

Våra främsta källor för patentsystemets historia är MacLeod \cite{macleod},
Bracha \cite{bracha} och Nard \cite{nard}.

Christine MacLeod sammanfattar brittiska patentsystemets utveckling mellan 
1600 och 1900 i rapporten \emph{Patents and Industrialisation - An Historical 
Overview of the British Case, 1624-1907}. Vi fann denna på brittiska 
Intellectual Property Offices hemsida och drar slutsatsen att detta är en 
rapport skriven för IPO. MacLeod är professor i historia vid University of 
Bristol i England varför vi finner rapporten trovärdig.

Oren Bracha ger en detaljrik och klarsynt bild av patentsystemets utveckling 
från monarkiska privilegierna i England i sin doktorsavhandling \emph{Owning 
Ides: A History of Angle-American Intellectual Property}. Bracha är nu 
juridikprofessor vid University of Texas School of Law. Avhandlingen citerar 
ofta ett annat av MacLeods verk så dessa två källor tar del av samma 
historieskrivning, vilket inte är så konstigt då få diskrepanser har upptäckts 
mellan olika källor.

Craig Allen Nard har skrivit boken \emph{The law of patents} som främst 
behandar juridiken bakom dagens patentsystem, men också viger ett kapitel åt 
patentsystemets uppkomst. Nard är juridikprofessor vid Case Western Reserve.

Dessa källor anses av oss vara trovärdiga då samtliga författare är erkända 
professorer inom sina fält och texterna i sig uppvisar hög akademisk standard. 
I alla de fall som dessa huvudkällor refererar vidare till andra verk för 
specifika detaljer har vi sökt att läsa originalkällan men i de fall dessa ej 
varit tillgängliga har vi litat på andrahandskällor.


\subsection{Avstamp}

Termen patent kommer från latinets \emph{litterae patentes} som
betyder öppet brev. Kopplingen till dagens patent är att medeltida monarker gav
ut 'privilegier', land, titlar, friheter och informerade om detta genom ett
öppet brev, \emph{litterae patentes}, där det kungliga sigillet placerades på
ett sådant sätt att brevet skulle kunna läsas utan att bryta sigillet (jmf: 
vanligt, stängt brev). Breven var också uttryckligen adresserade till 
vemhelst som läser det, alltså alla. Dessa privilegier kunde innefatta ensamrätt
att sälja en viss vara. Ordet patent har levt kvar genom historien till 
att nu innebära kommersiell ensamrätt till främst uppfinningar men även 
design etc. \cite{blackstone vad hänvisas?}

När vi talar om patent i historisk mening är det viktigt att poängtera att det 
snarare är \emph{litterae patentes} med koppling till de tidigare 
privilegierna än patent i modern bemärkelse. Första delen i rapporten beskriver
uppkomsten av vad vi idag skulle kalla patent i renässansens Italien.
Andra delen följer utvecklingen av patentsystemet i England under perioden
1500-1800. Rapportens tredje del går vidare till USA, och beskriver uppkomsten
av patentsystemet där, och hur det har förändrats med tiden. 

