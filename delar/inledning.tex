\begin{abstract}
	hej hopp
\end{abstract}	
	
	
\section{inledning}	
Introduktion av det problemområde eller den företeelse som skall studeras
Syfte med klart definierade frågeställningar
Kort metoddiskussion om källmaterialet

Våra främsta källor för patent systemets historia är MacLeod \cite{macleod}, Bracha \cite{bracha} och Nard \cite{nard}. Chrisitne MacLeod sammanfattar brittiska patentsystemets utveckling mellan 1600 och 1900 i rapporten \emph{Patents and Industrialisation - An Historical Overview of the British Case, 1624-1907}, vi fann denna på brittiska Intellectual Property Offices hemsida och drar slutsatsen att detta är en rapport skriven för IPO. MacLeod är historie professor vid University of Bristol i England varför vi finner rapporten trovärdig. Oren Bracha ger en detaljrik och klarsynt bild av patent systemets utveckling från monarkiska privilegierna i England i hans doktorsavhandling \emph{Owning Ides: A History of Angle-American Intellectual Property} Bracha är nu juridik proffesor vid University of Texas School of Law. Avhandlingen citerar ofta ett av annat av MacLeods verk så dessa två källor tar del av samma historieskrivning, vilket inte är så konstigt då få diskrepanser har upptäckts mellan olika källor. Craig Allen NArd har skrivit boken \emph{The law of patents} som främst behandar juridiken bakom patentsystemet idag men också viger ett kapitell åt patentsystemets uppkomst, Nard är juridik professor vid Case Western Reserve. Dessa källor anses trovärdiga då samtliga författare är erkända professorer inom sina fält och texterna i sig uppvisar hög akademisk standard. I alla de fall som dessa huvudkällor refererar vidare till andra verk för specifika detaljer har vi sökt att läsa orginalkällan men i de fall dessa ej varit tillgängliga har vi litat på andra hands källor.
  

Definition av patent.

termen patent kommer från latinets \emph{litterae patentes} som betyder öppet brev. Kopplingen till dagens patent är att medeltida monarker gav ut 'privilegier', land, titlar, friheter och informerade om detta genom ett öppet brev, \emph{litterae patentes}, där det kungliga sigillet placerades längst ned på brevet så att det skulle kunna läsas utan att bryta sigillet (jmf: vanligt, stängt brev). Dessa brev var också uttryckligen addreserade till vemhälst som läser det, alltså alla. Dessa privilegier kunde innefatta ensam rätt att säla en viss vara. Termen patent har levt kvar genom historien till att nu innebära en kommersiell ensamrätt till främst en uppfinning men även design etc. \cite{blackstone vad hänvisas?}

när vi talar om patent i historisk mening är det viktigt att poängtera att det snarare är \emph{litterae patentes} med koppling till de tidigare preivilegierna än moderna patatent. Första delen i rapporten beskriver vad vi idag skulle kalla patents härkomst ifrån de sentida \emph{litterae patentes}.

diskusion om källor, viktning, trovärdighet.

