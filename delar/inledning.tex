
\begin{abstract}
	hej hopp
\end{abstract}	
	
	
\section{Inledning}

Patent är en ensamrätt att utnyttja en ny uppfinning. Det innebär att ingen
annan får utnyttja uppfinningen genom att tillverka, sälja eller importera 
uppfinningen utan patenthavarens tillstånd. Patent kan anses vara en form av
upphovsrätt, men det är patent enligt den ovan nämnda beskrivningen som vi valt
att studera.

Introduktion av det problemområde eller den företeelse som skall studeras. 

Arbetets syfte är att utreda det historiska ursprunget till patentsystemet och
granska det kritiskt i ljus av den aktuella debatten kring patent.
De frågeställningar som ska besvaras av arbetet är: När började man utfärda
patent? Varför utfärdade man patent? Vilka alternativ fanns innan 
patent? Vilka har haft makt över, och vilka har gynnats av patentsystemet?

\subsection{Metod och källor}

Våra främsta källor för patent systemets historia är MacLeod \cite{macleod}, Bracha \cite{bracha} och Nard
\cite{nard}.

Christine MacLeod sammanfattar brittiska patentsystemets utveckling mellan 
1600 och 1900 i rapporten \emph{Patents and Industrialisation - An Historical 
Overview of the British Case, 1624-1907}, vi fann denna på brittiska 
Intellectual Property Offices hemsida och drar slutsatsen att detta är en 
rapport skriven för IPO. MacLeod är professor i historia vid University of 
Bristol i England varför vi finner rapporten trovärdig.

Oren Bracha ger en detaljrik och klarsynt bild av patentsystemets utveckling 
från monarkiska privilegierna i England i sin doktorsavhandling \emph{Owning 
Ides: A History of Angle-American Intellectual Property}. Bracha är nu 
juridikprofessor vid University of Texas School of Law. Avhandlingen citerar 
ofta ett annat av MacLeods verk så dessa två källor tar del av samma 
historieskrivning, vilket inte är så konstigt då få diskrepanser har upptäckts 
mellan olika källor.

Craig Allen Nard har skrivit boken \emph{The law of patents} som främst 
behandar juridiken bakom dagens patentsystem, men också viger ett kapitel åt 
patentsystemets uppkomst. Nard är juridikprofessor vid Case Western Reserve.

Dessa källor anses av oss vara trovärdiga då samtliga författare är erkända 
professorer inom sina fält och texterna i sig uppvisar hög akademisk standard. 
I alla de fall som dessa huvudkällor refererar vidare till andra verk för 
specifika detaljer har vi sökt att läsa orginalkällan men i de fall dessa ej 
varit tillgängliga har vi litat på andrahandskällor.

\paragraph{Termen patent} kommer från latinets \emph{litterae patentes} som betyder öppet 
brev. Kopplingen till dagens patent är att medeltida monarker gav ut
'privilegier', land, titlar, friheter och informerade om detta genom ett öppet 
brev, \emph{litterae patentes}, där det kungliga sigillet placerades på ett
sådant sätt att brevet skulle kunna läsas utan att bryta sigillet (jmf: 
vanligt, stängt brev). Breven var också uttryckligen addreserade till 
vemhelst som läser det, alltså alla. Dessa privilegier kunde innefatta ensamrätt
att sälja en viss vara. Ordet patent har levt kvar genom historien till 
att nu innebära kommersiell ensamrätt till främst uppfinningar men även 
design etc. \cite{blackstone vad hänvisas?}

När vi talar om patent i historisk mening är det viktigt att poängtera att det 
snarare är \emph{litterae patentes} med koppling till de tidigare 
privilegierna än patent i modern bemärkelse. Första delen i rapporten beskriver
härkomsten av vad vi idag skulle kalla patent, ifrån de sentida \emph{litterae patentes}.