\section{Europa}

\subsection{Innan patent systemet / privilegier och gillen} % (fold)
\label{sub:innan_patent_systemet}
*gillen mycket viktigare

Fram till och med mitten av medeltiden var härskares främsta metod för att attrahera främmande kompetens att ge handelsmän och hantverkare privilegier om de flyttade till härskarens domän. 
Dessa privilegier varierade vitt mellan olika härskare och beroende på hur viktig immigrationen ansågs och inkluderade allt ifrån subvention av verksamhet till skydd från gillen och monopolsrätt. 
Bruket att ge privilegier användes till viss del med goda avsikter, att främja den lokala produktionen och handeln men vissa privilegier gavs också till härskarens vänner för att styrka deras position gentemot deras konkurrenter.


Under medeltiden hade köpmän börjat skydda sina intressen genom att starta gillen. 
Dessa representerade de enskilda handelsmännen men kunde också reglera verksamheten hos individer. 
På grund av sin storlek, ofta alla hantverkare inom en disciplin till exempel textilvävning i en stad eller ett område, hade de ofta hela marknaden och då den enda konkurrensen kommer inifrån gillet uppstod ibland prisfixering. 
För att hindra detta fick härskare lura till sig utomstående kompetens från andra områden.

I medeltidens Europa saknades bra informationsflöde för handelsidéer annat än lokala handelsgillen och migration av yrkesverksamma individer. 
För att gynna den lokala handeln såg många härskare till att skapa en så förmånlig arbetssituation att handelsmän och hantverkare skulle vilja flytta dit. 
Detta genom att ge ut privilegier som skydd att bedriva verksamhet som annars reglerades av gillen, en subventionerad verkstad eller råvaror och ibland regelrätt monopol. 
<-- hampus blobsida. 
Dessa privilegier kunde även ges som belöning till personer som introducerade nya tekniker eller varor till området eller landet. 
Men även till härskarens vänner [exempelvis monopol på salt i england] varför befolkningen vände sig mot privilegiesystemet i stort.


Denna handelsinvandring kan anses vara en av anledningarna till att England utvecklades snabbt under 1300

% subsection innan_patent_systemet (end)

\subsection{steg mot ett patent system} % (fold)
\label{sub:steg_mot_ett_patent_system}
Under 1400-talet började ett konkret patentsystem utvecklas i södra europa. 
Detta värderade ny kunskap och nya uppfinningar som delgavs folket. 
I venedigs PAPER DIL PATENT-O omfattades många av stöttepelarna i dagens patentsystem. 


Historiens första patent gavs år 1421 av republiken Florens till Filippo Brunelleschi för vinch?systemet på det skepp han konstruerat för att frakta marmor till Doumo de Florence, en känd dom. 
Dock sjönk skeppet och några fler patent gavs ej ut. 
Men detta första steg levde kvar i textilgillenas praxis att ge ensamrätt till en hantverkare som designar ett nytt mönster eller vävnadsteknik. 


Senare, år 1474, nedtecknades i Venedig en patentlag som inspirerats av gillenas exklusivitetspraxis och -regler. 
Där fastslogs att om någon introducerar en uppfinnins skall denne få ensamrätt att producera och sälja sin upptäckt i tio år och vederbörandes ges laga beskydd mot intrång. 
I gengäld måste uppfinningen vara genuint ny och användbar för befolkningen. 
Det senare kravet tyder på att någon form av undersökning erfordrades innan ett patent utgavs. 
Lagen var skriven på den venetianska dialekten istället för de lärdas språk latin, vilket visar att den vände sig till hantverkare och handelsmän i första hand.


Det venetianska systemet spreds genom Euorpa i takt med att handelsmän och hantverkare emigrerade. 
(kunde upptas då gillen banat väg?) emmigrerar extramycket då venedig stgnerar

-> england


% subsection steg_mot_ett_patent_system (end)


