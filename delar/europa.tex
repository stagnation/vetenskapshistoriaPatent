\section{Hur patentsystemet växer fram i Europa}

\subsection{Innan patent -- privilegier, gillen och skrån} % (fold)
\label{sub:innan_patent_systemet}
%*gillen/skrån mycket viktigare?

Fram till och med mitten av medeltiden var härskares främsta medel för att
attrahera främmande kompetens att ge handelsmän och hantverkare privilegier i
utbyte mot att de flyttade till härskarens domän. 
Vad dessa privilegier innefattade varierade vitt mellan olika härskare och
beroende på hur viktig immigrationen ansågs; och kunde inkluderade allt ifrån
subvention av verksamheten till skydd från gillen, och monopolsrätt. 
Bruket att ge privilegier användes till viss del med goda avsikter, att främja
den lokala produktionen och handeln \cite{nard}. Men det hände att härskaren
missbrukade sin makt för att framhäva sina vänner gentemot kunkurrenter.

Under medeltiden hade köpmän börjat skydda sina intressen genom att starta
gillen. <-- gillen hampus?
Dessa representerade de enskilda handelsmännen men kunde också reglera
verksamheten hos individer.
På grund av sin storlek, ofta alla hantverkare inom en disciplin till exempel
textilvävning i en stad eller ett område, hade de ofta hela marknaden och då
den enda konkurrensen kommer inifrån gillet uppstod ibland prisfixering. 
För att motverka detta fick härskare locka till sig utomstående kompetens från
andra områden.

I medeltidens Europa saknades bra informationsflöde för handelsidéer annat än
inom lokala handelsgillen, och migration av yrkesverksamma individer. 
Dörför såg många härskare till att skapa en förmånlig arbetssituation för att
handelsmän och hantverkare skulle vilja flytta dit. 
Detta genom att ge ut privilegier som skydd att bedriva verksamhet som annars
reglerades av gillen, en subventionerad verkstad eller råvaror och ibland
regelrätt monopol. <-- hampus blobsida. 

Denna handelsinvandring kan anses vara en av anledningarna till att England
utvecklades snabbt under 1300



\subsection{Steg mot ett patentsystem} % (fold)
\label{sub:steg_mot_ett_patent_system}

Under 1400-talet började ett första patentsystem utvecklas i södra Europa.
Genom vilket man sökte skapa incitament för ny kunskap och nya uppfinningar att delges folket. 
I Venedigs PAPER DIL PATENT-O *namn? omfattades många av stöttepelarna i dagens patentsystem.

Historiens första patent utfärdades år 1421 av republiken Florens, till Filippo Brunelleschi för det skepp han konstruerat för att frakta marmor till Doumo de Florence, en känd dom \cite{frumkin}. 
Dock sjönk skeppet och några fler florentinska patent gavs ej ut.

Utveckling mot ett patentsystem fortsatte då Italienska textilskrån införde interna regler som gav ensamrätt till medlemmar för som kommit på nya mönster.
Gillen och hantverksskrån kontrollerade på den här tiden statsstaternas handel i stora delar av Italien och övriga Europa, men det är inte klart om gillenas inflytande drev på utvecklingen mot det första patentsystemet, eller om det skedde oberoende.

I republiken Venedig började man under mitten av 1400-talet att ge ut officiella patent och år 1474 stiftades den första av historien kända patentlagen\cite{frumkin}.
Den var skriven på den venetianska dialekten snarare än de lärdas språk, latin, vilket visar att den vände sig till hantverkare och handelsmän i första hand.
Lagen fastslog att om någon som introducerar en uppfinning ska få ensamrätt att producera och sälja sin upptäckt i tio år, och vederbörande garanteras laga skydd mot intrång. 
Kraven var i gengäld att uppfinningen skulle vara genuint ny inom väldet och användbar för befolkningen. 
Nard observerar att det senare kravet antyder att någon form av undersökning erfordrades\cite{nard}. 

Den venetianska patentlagen från 1474 besatt alla de egenskaper som ses som grundläggande för ett modernt patentsystem:

\begin{itemize}
    \item Uppfinnaren fick tidsbegränsad (tio år) ensamrätt i utbyte mot att uppfinningen avslöjades.
    \item Det ställdes krav på nyhet och användbarhet, vilket implicerar en granskningsprocess.
    \item Kraven var geografiskt avgränsade till republikens välde.
\end{itemize}

Idén med patent spreds genom Europa i takt med att venetianska hantverkare emigrerade. 
Innan 1600-talets början hade liknande patentsystem införts i flera europeiska länder, bland andra England.

% subsection steg_mot_ett_patent_system (end)
