\section{Hur patentsystemet växer fram i Europa}

\subsection{Innan patent systemet / privilegier och gillen} % (fold)
\label{sub:innan_patent_systemet}
%*gillen/skrån mycket viktigare?

Fram till och med mitten av medeltiden var härskares främsta medel för att
attrahera främmande kompetens att ge handelsmän och hantverkare privilegier i
utbyte mot att de flyttade till härskarens domän. 
Vad dessa privilegier innefattade varierade vitt mellan olika härskare och
beroende på hur viktig immigrationen ansågs; och kunde inkluderade allt ifrån
subvention av verksamheten till skydd från gillen, och monopolsrätt. 
Bruket att ge privilegier användes till viss del med goda avsikter, att främja
den lokala produktionen och handeln \cite{nard}. Men det hände att härskaren
missbrukade sin makt för att framhäva sina vänner gentemot kunkurrenter.

Under medeltiden hade köpmän börjat skydda sina intressen genom att starta
gillen. <-- gillen hampus?
Dessa representerade de enskilda handelsmännen men kunde också reglera
verksamheten hos individer.
På grund av sin storlek, ofta alla hantverkare inom en disciplin till exempel
textilvävning i en stad eller ett område, hade de ofta hela marknaden och då
den enda konkurrensen kommer inifrån gillet uppstod ibland prisfixering. 
För att motverka detta fick härskare lura till sig utomstående kompetens från
andra områden.

I medeltidens Europa saknades bra informationsflöde för handelsidéer annat än
inom lokala handelsgillen, och migration av yrkesverksamma individer. 
Dörför såg många härskare till att skapa en förmånlig arbetssituation för att
handelsmän och hantverkare skulle vilja flytta dit. 
Detta genom att ge ut privilegier som skydd att bedriva verksamhet som annars
reglerades av gillen, en subventionerad verkstad eller råvaror och ibland
regelrätt monopol. <-- hampus blobsida. 

Denna handelsinvandring kan anses vara en av anledningarna till att England
utvecklades snabbt under 1300


\subsection{Steg mot ett patentsystem}

Under 1400-talet började ett första patentsystem utvecklas i södra europa. 
Detta värderade ny kunskap och nya uppfinningar som delgavs folket. 
I venedigs PAPER DIL PATENT-O omfattades många av stöttepelarna i dagens patentsystem. 


Historiens första patent gavs år 1421 av republiken Florens till Filippo Brunelleschi för det skepp han konstruerat för att frakta marmor till Doumo de Florence, en känd dom \cite{frumkin}. 
Dock sjönk skeppet och några fler patent gavs ej ut. 

Även handelsgillena hade ett exklusivitetssystem för uppfinning av ny design under den här perioden. Det är inte klart om gillenas interna praxis direkt kopierades från stadsstaternas förstadier till patentsystem eller om dessa utvecklades parallellet. Nard observerar att ``The Italian textile guilds, reflecting the growth of commercial activity, filled the void, enacting private rules granting exclusive rights to those members of the guild who invented ‘certain . . . designs and patterns' of silk or wool.'' \cite{nard}. *citerar annan källa....

*relation mellan gillen o patent, samtida utveckling eller efterföljning?

Under 1400-talet hade även Vendig sneglat på ett patentsystem och efter 1450 börjades patent ges ut av officilas och år 1474 nedtecknades en konkret patentlag\cite{frumkin}. 
Där fastslogs att om någon introducerar en uppfinnins skall denne få ensamrätt att producera och sälja sin upptäckt i tio år och vederbörandes ges laga beskydd mot intrång. 
I gengäld måste uppfinningen vara genuint ny och användbar för befolkningen. 
Nard observerat att Det senare kravet tyder på att någon form av undersökning erfordrades innan ett patent utgavs\cite{nard}. 
Lagen var skriven på den venetianska dialekten istället för de lärdas språk latin, vilket visar att den vände sig till hantverkare och handelsmän i första hand.

Det venetianska systemet spreds genom Euorpa i takt med att handelsmän och hantverkare emigrerade. 
(****kunde upptas då gillen banat väg?) emmigrerar extramycket då venedig stgnerar\cite{nard}.

-> england


