\section{Immaterialrätt}

Innan pålitliga patentsystem inrättats stod man som uppfinnare inför ett problem, om man ville förverkliga sin upptäckt.
En uppfinnare kan inte få finansiering utan att avslöja idén, men så fort man gjorde det fanns det inget skydd ifrån att finansiären själv skulle realisera idén och lämna uppfinnaren tomhänt \cite{nard}.
Detta problem är känt som \emph{Arrows informationsparadox} och löstes inte förrän införandet av immaterialrätt, \emph{intellectual property rights}.

I en global marknad där immateriella tillgångar, \emph{intellectual property}, fått allt starkare ställning blir patenthantering en viktig del av företagens verksamhet.
Patent är nu strategiska medel som krävs för att ett företag skall kunna äntra marknaden.
För att kunna utnyttja en teknologi måste man ha alla relevanta rättigheter.
En möjlighet är att köpa patent av ett annat företag för att låsa upp marknaden.
Detta ger ytterliggare en anledning att ta patent, någon annan kan vilja köpa det i ett senare skede.

Patent kan också användas för att stävja utveckling hos konkurrenter.
Genom att utnyttja patentstrategier som sätter hinder eller skärmar av områden i \emph{teknikrymden}, kan utveckling på ett teknikområde göras dyrare eller blockeras för konkurrenter \cite{ove}.

En form av innovation som möjliggörs av patentsystemet är s.k. \emph{öppen innovation}.
Konceptet innebär att olika aktörer kan stå för uppfinnande respektive kommersialisering, för att på så vis skapa innovation.
De två former som tydligast illustrerar koncepter kallas \emph{inåtgående} (\emph{inbound}) respektive \emph{utåtgående} (\emph{outbound}) öppen innovation, där namnen indikerar riktningen på idéflödet.
Inåtgående öppen innovation praktiseras genom att externa idéer förvärvas och kommersialiseras, medan motsatsen gäller för utåtgående öppen innovation, interna idéer överlåts till andra för att kommersialisera.
Andra former existera också i med gemensamma insatser och samriskföretag.
Gemensamt för alla former av öppen innovation är överföringar av idéer, vilket gör dem beroende av ett fungerande immaterialrättssystem.
Öppen innnovation öppnar för företag att specialisera och fokusera sin verksamhet mer effektivt.
Köparen av en teknologi minskar kostnaden och risken förknippad med utvecklingsinsatsen. 
Medan företag som specialiserar på utveckling kan profitera på att sälja sina upptäckter.
