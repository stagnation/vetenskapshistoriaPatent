\section{USA} 
*first to invent?

\subsection{Kolonier med inspiration från England}

Det tidiga amerikanska patentsystem utgick ifrån det Engelska systemet. Innan Staternas enande fanns det
flera kolonier som gav ut patent enligt engelsk privilegiemodell, en konkret undersökning av varje
patentansökan samt dessa för- och nackdelar för befolkningen i kolonin eller staten. För ett eviljat
patent kunde man få pengar, skattelättnad eller ett monopol för introduktion av en marknad. Bracha
observerar att Amerikanarna var långsammare än britterna med skiftet från marknadsnisch till
uppfinning\cite{bracha}. I det här skedet användes ofta \emph{working clause} och \emph{apprentice
clause} men en specifikation var ett undantag \cite{bracha}.

Bracha menar att kolonin / staten ansågs sig ha en rätt och skyldighet att aktivt ingripa i marknaden
för för att gynna almänheten\cite{bracha}. Patent användes således som ekonomiska styrmedel likt de
kungliga privilegierna i England. En viktig del i patentsystemets framväxt i England var tvisten om makt
mellan kungen som gav ut privilegier och regeringen men i USA fanns bara en part, kolonins "regering"
som också gav ut patenten, varför liten utbeckling skedde under 1600 och 1700 i USA.

*i england när kungen inte brydde sig godkändes alla patent, i usa när ingen brydde sig godkändes inga
patent.

???????Amerikanska patent landskapet började i mycket som tabula rasa, det fanns inga patent som gällde.
utan det blev lite vlda västern med att ta patent?????????? 
*galna privilegie patent till inemska
snubbar för andras uppfinnnigengar ??

\subsection{Oberoende och Enade USAs patent system}

USA skulle efter självständighetskriget att skriva en ny lagstifning och samtidigt inleddes en process
mot ett nytt patent system. Från slutet av 1700-talet pågick en utveckling från det gamla privilegie
systemet till det system vi känner till idag. ??????????

Efter självständigheten så fortsatte staterna i stort att ge ut patent enligt samma devis som tidigare i
kolonier. Dock blev dessa sprida system en komplikation i och med industrialisering och ökad handel över
hela landet. Många patent från olika stater kolliderade och man insåg att en enhetlig federal
lagstifning krävdes. Denna baserades i stort på den samtida brittiska föregångaren med uppfinningspatent
snarare än entreprenörsprivilegie \cite{nard}.

Den första patent acten i usa infördes år 1784 South Carolina. Där fastslods att en uppfinnare av
användbar maskin ska ha ensamrätt för 14 år med samma ställning som en bokförfattare\cite{bracha}. Detta var ett viktigt steg men mest symboliskt. Till viss mån gavs patent ut enligt
denna nya devis men fortfarande som ett privilegie efter särskild bedömning och samtidigt gavs den gamla
sortens patent ut parallellt \cite{bracha}.

*relatera till författarrätt -verkar vara mer utvecklad vid tidpunkten.

I (5) september år 1787 röstade kongressen igenom första steget mot ett nationellt lagstiftat
patentsystemet. De införde Artikel 1, Sektion 8, Punkt 8 av författningen, som säger att kongressen har
makt att "promote the Progress of Science and useful Arts by securing for limited Times to Authors and
Inventors the exclusive Right to their respective Writings and Discoveries." där Nard belyser att
Science syftar till skrivna verk samt upphovsrätt och useful Arts innefattar uppfiningar och metoder,
sådant som kan patenteras i dagens märkning\cite{nard}. Ett viktigt steg ifrån entreprenörsprivilegier.

Patentbegreppet vidareutvecklades senare år 1790 när the Patent Act infördes av George Washington. Nu
kunde man få patent för en upptäckt, vidareutveckling eller uppfinning genom att ansöka till ett
specifikt organ som kontrollerade att uppfinningarna var tillräckligt användbara\cite{nard}. Detta
granskande organ hade endast tre poster, Secretary of State, Secretary of War och Attorney General, och
var kraftigt underdimensioneat. Dessa tre poster hade dessutom andra pressande ansvarsområden varför
endast 55 patent beviljades under denna Act\cite{nard}. I det här skedet var patenttiden 14 år, vilket
mötes av kritik då kommersialiseringen av en uppfinning kunde ta flera år. Endast amerikanska
medbordgare tilläts ta patent\cite{nard}. <- nard här?

koppling till plagiat av patent från annanstäders? behövdes en model? endast konkreta ideér?

För att lösa detta uppdaterades systemet genom att ersätta kongressorganet med ett registrations kontor
av tjänstemän redan tre år senare, 1793. Detta var ett försök att underlätta och smidiggöra
ansökansprocessen, som hittils varit väldigt lång. Nu lämnades en ansökan in till the Secretary of State
som bad Attorney General om undersökning. Dock blev det lite för smidigt och kravet att en uppfinning
skulle vara "tillräckligt användbar" släpptes, utan det räckte att vara "användbar", eller "användbar i
en uppkommande marknad". Dessa vaga krav och bristfällig kontroll av patentansökningar ledde till att en
uppsjö av tramsiga patent godkändes\cite{nard}.

Runt sekelskitftet 1800 växte amerikanska ekonomin kraftig och såg ett starkt skifte från agrikultur
till industrialiserad produktion. I samma takt ökade handeln över hela landet och det blev tydligt att 1793-års lagstifning var lite för vag i kraven. Och många patent som inte var tillräckligt nya godkändes likväl\cite{nard}. Mr Ruggles belyser detta i en rapport till senaten, han säger att lagstifningen gav upphov till "unrestrained and promiscuous grants of patent privileges’" \cite{ruggles} år 1836.

Samma år reviderades reviderades patentsystemet igen, nu genom att införa det Patent
Office(? eller bara numreringen) som finns än idag. Nu med ökad fokus på revision av patentförslag innan
det godkändes, det infördes också en juridisk kanal att överklaga om patenten blev avfärdade \cite{nard}.

-> Weber, Gustavus A. The Patent Office: Its History, Activities and Organization:

En viktig del av nya Acten var att den införde ett system där nya patent exponerades för folket i stort
på bibliotek i hela landet så att aptenttagare enklare kunde kontrollera att deras upptäckt verkligen
var ny\cite{watson}. Detta var ytterliggare ett steg närmre idén att patent ges till uppfinnare/ i
utbyte mot att de ger kunskap till folket. Vidare möjligjordes utökning av patentskyddet ytterligare sju
år \cite{watson} samt kravet på amerikansk nationalitet hos patentsökaren släpptes.

Transparansen i patenten stärktes ytterliggare när the Patent Act of 1870 lade tyngre vikt på ett
patents specifikation. Från och med nu levde och föll patent på hur specifikationen var formulerad, som
bieffekt spreds mer kunskap till folket i och med tydligare förklaringar\cite{nard}.


\subsection{Modern historia, 1900-tal} 

Under slutet av 1800-talet kom patent att återförknippas med monopol eftersom de användes
restriktivt under depressionen  på 1890-talet. Till följd infördes både lagar och domstolspraxis mot 'monopolsmissbruk' och andra ekonomiska metoder att utesluta aktörer fårn marknaden. Ett steg var att införa Sherman Antitrust Act men dommarna oberopade också tvivelaktiga

 Med början runt 1930 blev domstolarna allt mer negativt inställd atill
patent. Praxis att anberoppa ett \emph{flash of genius} blev vidspridd, i synnerhet efter fallet Cuno
Engineering v. Automatic Devices (1941). Ett krav stäldes att endast upptäckter som genomförts i ett
genialt ögonblick skulle upphållas -således föll alla patent som bygde på succesiv förbättring och
patentet måste tydligt uppvisa denna härkomst\cite{nard}. 

Patent Misuse Doctrine, som säger att om ett patent är för vidt, till den grad att den hämar legitim
kokurrens kan patentägaren ej stämma någon för intrång, användes allt oftare som försvar i rättsfall, i
synnerhet fallet Mercoid Corp. v. Mid-Continent Inv. Co.,\cite{nard}. Detta och annan praxis under
perioden gjorde det svårt att upprätthålla patent och precis som i England under sekelskifftet 17,1800
blev det viktigt att se över lagstiftningen och konkretisera vad ett patent är och hur det ska behandlas.

Det är för oss inte klart vad att hämma legitim konkurrens innebär och 

Först 1952 återuppbyggdes patentsystemet och grunden till den moderna tolkningen lades. Nu infördes
starkare krav på användbarhet, novelitet och icke-uppenbarhet. Det krävdes också en förklaring av vad
intrång till patentet innebär i ansökan. Till följd av detta tonades Patent Misuse Doctrine ned och
kravet på \emph{a flash of genious} upplöstes helt\cite{nard}.

Dock var patentmål fortfarande lösta i regionala domstolar och det fanns stora skillnader i hur olika
domstolar dömde i patentmål\cite{nard}. Det blev väldigt viktigt för båda pater att hålla målet i en
domstol som delade deras synpunkter på patentsystemet. Detta löstes först 1982 när the United States
Court of Appeals of the Federal Circuit skapades för att ta upp alla patentmål\cite{nard}. Instifftandet
av de nya domstolen gjorde patentlandskapet säkrare och stärkte patentens ställning. Studier visar att
efter instiftandet har patent ogiltigförklarats mindre ofta och fler sådana ogiltigförklaringar från
Regional Court har ogillats vid överlagan till FC. \cite{henry} Detta är relevant då patentmål i domstol
ofta handlar om varesig patentet ifråga, som målsägare anses har inträngts, är giltigt överhuvudtaget.
Om domstolen beslutar att patentet är gilltigt så behandlas eventuellt patentintrång. Då fler patent
fanns giltiga efter CAFC kunde fler dömmas för patentintrång även om andelen som döms för intrång av
gilltigt patent ej ändrats särskilt mycket efter domstolens instifning. 

Om detta seda när bra är en
annan fråga. Vissa har dock höjt rösten ... nard

Stärkt validitet av patent är viktigt då risken att patentet avfärdas nu är lägre vid stämning för
intrång har patentägaren mindre att förlora (jmf: hela patentet) i att stämma någon för intrång. men
äveon om patentet erkänns som giltigt kan domstolen dömma att den alleged inträngaren inte intränger,
har då satt end foten på vad patentet egentligen skyddar, och detta kan i praktiken göra patentet
betydelselöst om domstolen finner att patentets skyddsområde är litet. andra kan gå i den frikända
inträngarens fotsteg utan efterdyninigar. även detta från \cite{henry}

 mer incitament till att ta patent och moderna begreppet patentportfolio. 

