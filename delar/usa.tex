\section{Patentsystemets utveckling i USA} 
\label{sec:usa}

\subsection{Kolonier med inspiration från England}

Det tidiga amerikanska patentsystem utgick ifrån det Engelska systemet. Innan Staternas enande fanns det
flera kolonier som gav ut patent enligt engelsk privilegiemodell, en konkret undersökning av varje
patentansökan samt dessa för- och nackdelar för befolkningen i kolonin eller staten. För ett beviljat
patent kunde man få pengar, skattelättnad eller ett monopol för introduktion av en marknad. Bracha
observerar att Amerikanarna var långsammare än britterna med skiftet från marknadsnisch till
uppfinning\cite{bracha}. I det här skedet användes ofta \emph{working clause} och \emph{apprentice
clause} men en specifikation var ett undantag \cite{bracha}.

Bracha menar att kolonin / staten ansågs sig ha en rätt och skyldighet att aktivt ingripa i marknaden
för för att gynna almänheten\cite{bracha}. Patent användes således som ekonomiska styrmedel likt de
kungliga privilegierna i England. En viktig del i patentsystemets framväxt i England var tvisten om makt
mellan kungen som gav ut privilegier och regeringen men i USA fanns bara en part, kolonins "regering"
som också gav ut patenten, varför liten utveckling skedde under 1600 och 1700 i USA.

%*i england när kungen inte brydde sig godkändes alla patent, i usa när ingen brydde sig godkändes inga
%patent.

%???????Amerikanska patent landskapet började i mycket som tabula rasa, det fanns inga patent som gällde.
%utan det blev lite vlda västern med att ta patent?????????? 
%*galna privilegie patent till inemska
%snubbar för andras uppfinnnigengar ??
%*importpatent?

\subsection{Oberoende och Enade USAs patent system}

USA skulle efter självständighetskriget skriva en ny lagstifning och samtidigt inleddes en process
mot ett nytt patentsystem. Från slutet av 1700-talet pågick en utveckling från det gamla privilegiesystemet till det system vi känner till idag.
????

Direkt efter självständigheten så fortsatte staterna i stort att ge ut patent enligt samma devis som tidigare i
kolonierna. Dock blev dessa sprida system en komplikation i och med industrialisering och ökad handel över
hela landet. Många patent från olika stater kolliderade och man insåg att en enhetlig federal
lagstifning krävdes. Denna baserades i stort på den samtida brittiska föregångaren med uppfinningspatent
snarare än entreprenörsprivilegie \cite{nard}.

Den första patent acten i usa infördes år 1784 South Carolina. Där fastslods att en uppfinnare av
användbar maskin ska ha ensamrätt för 14 år med samma ställning som en bokförfattare\cite{bracha}. Detta var ett viktigt steg men mest symboliskt. Till viss mån gavs patent ut enligt
denna nya devis men fortfarande som ett privilegie efter särskild bedömning och samtidigt gavs den gamla
sortens patent ut parallellt \cite{bracha}.

%*relatera till författarrätt -verkar vara mer utvecklad vid tidpunkten.


\subsection{Viktig lagstifning i USAs tidiga patentsystem}

Den 5:e september år 1787 röstade kongressen igenom de förenade staternas författning. Införandet av Artikel 1, Sektion 8, Punkt 8 av författningen skulle bli första steget mot ett nationellt lagstiftat patentsystem. Punkten säger att kongressen har
makt att "promote the Progress of Science and useful Arts by securing for limited Times to Authors and
Inventors the exclusive Right to their respective Writings and Discoveries.". Där Nard belyser att
Science syftar till skrivna verk samt upphovsrätt och useful Arts innefattar uppfiningar och metoder,
sådant som kan patenteras i dagens märkning\cite{nard}. 

%*vad betyder inventor i konst? *kan argumentera att om inv. är uppfinnare så står det first to invent de jure i konst. om dt är entreprenörs tolkning öppet för first to file <- någon blog ej tillräckligt tillförlitligt, men sunt resonemang.

%*Bracha menar att "inventor" i författningen delar den moderna konnotationen av en uppfinnare eller någon som upptäckt en process.

Men Baracha kommenterar att även om den skrivna lagen indikerade ett mer modernt patentsystem så såg aktörerna ingen anledning att gå ifrån det gamla privilegiesystemet då den nya lagstiftningen var tillräckligt kompatibel \cite{bracha}. Det vill säga man sökte patent för främst uppfinningar men patenten beviljades som privilegier, där patentsökare ofta beskrev fördelar för allmänheten skulle den godkänna hans ansökan \cite{bracha}. %*förtydliga privilegier

Patentbegreppet vidareutvecklades senare år 1790 när the Patent Act infördes av George Washington. Nu
kunde man få patent för en upptäckt, vidareutveckling eller uppfinning genom att ansöka till ett
specifikt organ som kontrollerade att uppfinningarna var tillräckligt användbara\cite{nard}. I det här skedet var patenttiden 14 år, vilket mötes av kritik då kommersialiseringen 
 av en uppfinning kunde ta flera år. Endast amerikanska medbordgare tilläts ta patent\cite{nard}.  

Det granskande organet hade endast tre poster, Secretary of State, Secretary of War och Attorney General\cite{bracha}, och
var kraftigt underdimensionerat. Dessa tre poster hade dessutom andra pressande ansvarsområden varför
och en grundlig undersökning av varje ansökan samt beslut om individuella privilegier var högst krävande varför endast 55 patent beviljades under denna Act\cite{nard}.  

Denna överbelastning löstes genom att ersätta kongressorganet med ett registrationskontor
av tjänstemän redan tre år senare, 1793. Detta var ett försök att underlätta och smidiggöra
ansökansprocessen, som hittils varit väldigt lång. Nu lämnades en ansökan in till the Secretary of State
som bad Attorney General om undersökning. Dock blev det lite för smidigt och kravet att en uppfinning
skulle vara "tillräckligt användbar" släpptes, utan det räckte att vara "användbar", eller "användbar i
en uppkommande marknad". Dessa vaga krav och bristfällig kontroll av patentansökningar ledde till att en
uppsjö av tvivelaktiga patent godkändes\cite{nard}. 

Det är också viktigt att poängtera att amerikanska lagstfiningigen ej explicit berörde importpatent, vilket stod i kontrast till england där man fortfarande kunde ta patent för att enbart importera en uppfinning till england\cite{bracha}. Patent för importerade upptäckter hade behandlats i utkast till 1790-års lagstfitning men ströks innan genomröstning, Bracha menar att detta tyder på att importpatent lyftes som en fråga och ett beslut att ej tillåta dessa fattades\cite{bracha}. Varför man ej tillåts ta patent för importerade upptäckter. Ett annat perspektiv är att det redan fanns någon annan som saken i fråga, och denne skulle respekteras enligt "first and true inventor" i konstitutionen. Detta skulle i senare lagstifning konkretiseras som \emph{first to invent} prioriteringen. 

Runt sekelskitftet 1800 växte amerikanska ekonomin kraftig och såg ett starkt skifte från agrikultur
till industrialiserad produktion. I samma takt ökade handeln över hela landet och det blev tydligt att 1793-års lagstifning var för vag i kraven. Därför godkändes många patent som inte var tillräckligt nya \cite{nard}. Under denna tid diskuterades ofta varesig ett patent var användbart eller ej i domstol. Vissa menade att staten ej skulle undersöka om ett patent var användbart utan låta marknaden reglera detta, genom att endast användbara uppfinningar finner ekonomisk framgång \cite{bracha}.Detta kan ses som  ett steg mot rättighetsperspektivt, uppfinnaren har rätt till patent för alla nya uppfinnigar och sedan får marknaden avgöra hur stor belöning skall ges. Till skillnad från att staten ger privilegier som motsvarar den tänkta nyttan hos uppfinningen enligt privileigeperspektivet.

%*användbarhets diskusion

Mr Ruggles belyser lagens tillkortakommanden i en rapport till senaten, han säger att 1793-års lagstifning gav upphov till "unrestrained and promiscuous grants of patent privileges’" \cite{ruggles} år 1836. 
Detta år reviderades patentsystemet igen, nu genom att införa det Patent
Office som finns än idag\cite{nard} vilket skulle bli slutet för privilegiemodellen\cite{bracha}. Dennna act var mer anpassad för nationell handel och ett industrialliserat samhälle. Ökat fokus lades på revision av patentförslag innan
de godkändes mot en standard model för vad som var patenterbart istället för en individuell bestämmelse\cite{bracha}. Det infördes också en juridisk kanal att överklaga om patenten blev avfärdade \cite{nard}. Detta var också ett första distinkt steg mot att se patent som en rättighet för uppfinnare, mycket som copyright\cite{bracha}. Nu skulle en uppfinnare belönas proportionellt mot uppfinningens värde genom ensamrätt på marknaden där en bra uppfinning premierades.

%*-> Weber, Gustavus A. The Patent Office: Its History, Activities and Organization:

En viktig del av nya Acten var att den införde ett system där nya patent exponerades för folket i stort
på bibliotek i hela landet så att patenttagare enklare kunde kontrollera att deras upptäckt verkligen
var ny\cite{watson}. Detta var ytterliggare ett steg närmre idén att patent ges till uppfinnare/ i
utbyte mot att de ger kunskap till folket. Vidare möjligjordes utökning av patentskyddet ytterligare sju
år \cite{watson} dessutom läpptes kravet på amerikansk nationalitet hos patentsökaren.

Transparansen i patenten stärktes ytterliggare när the Patent Act of 1870 lade tyngre vikt på ett
patents specifikation. Från och med nu levde och föll patent på hur specifikationen var formulerad, som
bieffekt spreds mer kunskap till folket i och med tydligare förklaringar\cite{nard}.
