\section{USA} % (fold) \label{sec:usa}

\subsection{inspiration från England och finna sitt fotstöd,- footing} % (fold)
\label{sub:inspiration_från_england}

Amerikanska patentsystem utgick ifrån det Engelska systemet men gjordes även vitkiga förbättringar. Innan
Staternas enande fanns det flera stater som gav ut patent enligt engelsk modell, dock blev detta en
komplikation i och med industrialisering och ökad handel över hela landet. Många patent från olika stater
kolliderade och man insåg att en enhetlig federal lagstifning krävdes. Denna baserades i stort på den
brittiska föregångaren.

I (5) september år 1787 röstade kongressen igenom första steget mot ett lagstiftat patentsystemet. De
införde Artikel 1, Sektion 8, Klausul 8 av konstitutionen, som säger att kongressen har makt att "promote
the Progress of Science and useful Arts by securing for limited Times to Authors and Inventors the
exclusive Right to their respective Writings and Discoveries." där Science ankyts till skrivna verk,
upphovsrätt och useful Arts innefattar uppfiningar och metoder, sådant som kan patenteras i dagens märkning

Patentbegreppet vidareutvecklades senare år 1790 när the Patent Act infördes av George Washington. Nu kunde
man få patent för en upptäckt, vidareutveckling eller uppfinning genom att ansöka till ett specifikt organ
av kongressen(?) som kontrollerade at uppfinningarna var tillräckligt användbara. Detta var ett steg i rätt
rikting men ett granskande organ med endast tre poster, Secretary of State, Secretary of War och Attorney
General, var kraftigt underdimensioneat. Dessa tre poster hade andra pressande ansvarsområden varför endast
55 patent beviljades under denna Act. I det här skedet var patenttiden 14 år, vilket mötes av kritik då
kommersialiseringen av en uppfinning kunde ta flera år. Endast amerikanska medbordgare tilläts ta patent.

koppling till plagiat av patent från annanstäders?

För att lösa detta uppdaterades systemet genom att ersätta kongressorganet med ett registrations kontor av
tjänstemän r edan tre år senare. Detta var ett försök att underlätta och smidiggöra ansökansprocessen, som
hittils varit väldigt lång. Nu lämnades en ansökan in till the Secretary of State som bad Attorney General
om undersökning. Dock gjordes det lite för smidigt och kravet att en uppfinning skulle vara "tillräckligt
användbar" släpptes, utan det räckte att vara "användbar", eller "användbar i en uppkommande marknad".
Dessa vaga krav och bristfällig kontroll av patentansökningar ledde till att en uppsjö av tramsiga patent
godkändes.

Runt sekelskitftet 1800 växte amerikanska ekonomin kraftig och såg ett starkt skifte från agrikultur till
industrialiserad produktion. I samma takt ökade handeln över hela landet och 1793-års lagstifning blev
snabbt utdaterad då

 Det skulle dröja ända till 1836 innan systemet åter reviderades, nu genom att införa det Patent Office(?
eller bara numreringen) som finns än idag. Nu med ökad fokus på revision av patentförslag innan det
godkändes, det infördes också en juridisk kanal att överklaga om patenten blev avfärdade.

-> Weber, Gustavus A. The Patent Office: Its History, Activities and Organization:

En viktig del av nya Acten var att den införde ett system där nya patent exponerades för folket i stort på
bibliotek i hela landet så att aptenttagare enklare kunde kontrollera att deras upptäckt verkligen var ny.
Detta var ett steg närmre idén att patent ges till uppfinnare/ i utbyte mot att de ger kunskap till folket.
Vidare möjligjordes utökning av patentskyddet ytterligare sju år samt kravet på amerikansk nationalitet hos
patentsökaren släpptes.

Transparansen i patenten stärktes ytterliggare när the Patent Act of 1870 lade tyngre vikt på ett patents
specifikation. Från och med nu levde och föll patent på hur specifikationen var formulerad, som bieffekt
spreds mer kunskap till folket i och med tydligare förklaringar.

???????Amerikanska patent landskapet började i mycket som tabula rasa, det fanns inga patent som gällde.
utan det blev lite vlda västern med att ta patent???????????

% subsection inspiration_från_england (end)

\subsection{Modern historia, 1900-tal} % (fold) \label{sub:modern_historia_1900_tal}

Under slutet av 1800-talet kom patent att återförknippas med monopol i och med hur de användes restriktivt
under depressionen 1890. Varför the Sherman Antitrust Act instifftades. Denna förmedlade att handlingar hos
företag som "förstör konkurrens" är olagliga. ... OVAN:wikipedia säger negativt 1890-1952; nard säger
positiv 1890-1930 negativ 1930-1950

Motståndet till patent levde kvar efter att ekonomin kom på rätsida och förstärktes ytterliggare av
depressionen efter första världskriget. Med början runt 1930 blev domstolarna allt mer negativt inställda
till patent. Praxis att anberoppa ett \emph{flash of genius} blev vidspridd, i synnerhet efter fallet Cuno
Engineering v. Automatic Devices (1941). Ett krav stäldes att endast upptäckter som genomförts i ett
genialt ögonblick skulle upphållas -således föll alla patent som bygde på succesiv förbättring och patentet
måste tydligt uppvisa denna härkomst.

Patent Misuse Doctrine, som säger att om ett patent är för vidt, till den grad att den hämar legitim
kokurrens kan patentägaren ej stämma någon för intrång, användes allt oftare som försvar i rättsfall, i
synnerhet fallet [mercoid]. Detta och annan praxis under perioden gjorde det svårt att upprätthålla patent
och precis som i England under sekelskifftet 17,1800 blev det viktigt att se över lagstiftningen och
konkretisera vad ett patent är och hur det ska behandlas.

Först 1952 återuppbyggdes patentsystemet och grunden till den moderna tolkningen lades. Nu infördes
starkare krav på användbarhet, novelitet och icke-uppenbarhet. Det krävdes också en förklaring av vad
intrång till patentet innebär i ansökan. Till följd av detta tonades Patent Misuse Doctrine ned och kravet
på a flash of genious upplöstes helt.

Dock var patentmål fortfarande lösta i regionala domstolar och det fanns stor diskrepans i hur olika
domstolar dömde i patentmål. Det blev väldigt viktigt för båda pater att hålla målet i en domstol som
delade deras synpunkter på patentsystemet. Detta löstes först 1982 när the United States Court of Appeals
of the Federal Circuit skapades för att ta upp alla patentmål(alla?). Instifftandet av de nya domstolen
gjorde patentlandskapet säkrare och stärkte patentens ställning. Studier visar att efter instiftandet har
patent ogiltigförklarats mindre ofta och fler sådana ogiltigförklaringar från Regional Court har ogillats
vid överlagan till FC. \cite{henry} Detta är relevant då patentmål i domstol ofta handlar om varesig
patentet ifråga, som målsägare anses har inträngts, är giltigt överhuvudtaget. Om domstolen beslutar att
patentet är gilltigt så behandlas eventuellt patentintrång. Då fler patent fanns giltiga efter CAFC kunde
fler dömmas för patentintrång även om andelen som döms för intrång av gilltigt patent ej ändrats särskilt
mycket efter domstolens instifning. Om detta seda när bra är en annan fråga. ber byråkrati i patentsystemet

Stärkt validitet av patent är viktigt då risken att patentet avfärdas nu är lägre vid stämning för intrång har patentägaren mindre att förlora (jmf: hela patentet) i att stämma någon för intrång. men äveon om patentet erkänns som giltigt kan domstolen dömma att den alleged inträngaren inte intränger, har då satt end foten på vad patentet egentligen skyddar, och detta kan i praktiken göra patentet betydelselöst om domstolen finner att patentets skyddsområde är litet. andra kan gå i den frikända inträngarens fotsteg utan efterdyninigar. även detta från \cite{henry}

 mer incitament till att ta patent och moderna begreppet patentportfolio.
% subsection modern_historia_1900_tal (end)

% section usa (end)
