\subsection{Dagens patentsystem i USA}

I USA innebär ett patent att innehavaren kan \emph{exkludera} andra aktörer från marknaden. Mer specifikt kan denne förhindra andra från att \emph{tillverka}, \emph{använda} eller \emph{sälja} det som skyddats. Sedan internationella patentregler infördes 1996 gäller exklusiviteten även att \emph{erbjuda att sälja} samt att \emph{importera} \cite{cmu-overview}.

% vem får söka
Man skiljer på att komma på en idé och att omsätta idéer till praktik. Det är endast de som kommer på idéer som räknas som uppfinnare. Därtill måste man vara själva uppfinnaren för att få patent; ett företag som anställer någon kan inte söka dennes patent, inte heller någon som finansierar forskningen. Uppfinnaren signerar patentet tillsammans med ett löfte på heder och samvete om att man är den sanne uppfinnaren \cite{cmu-overview}.

Alla som står med på patentansökan måste vara verkliga uppfinnare, och därtill måste alla verkliga uppfinnare stå med på patentansökan.

% vad kan man söka
Man skiljer på tre olika typer av patent. Den vanligaste och mest kända typen av patent är \emph{utility patent} \cite{cmu-overview}.
\todo{Referera här istället till listan av de generella kraven på patent.}
% Kraven är att det som patenteras är \emph{nytt}, \emph{användbart} och \emph{ej är uppenbart}. 
Därtill finns \emph{växtpatent} på växter som man skapat asexuellt samt designpatent som rör utseende men ej funktion eller struktur. Designpatent är i praktiken mycket smala, och förhindrar bara rena kopior. Den maximala skyddstiden för designpatent är 14 år; densamma för de övriga typerna av patent ökade år 1996 från 17 till 20 år.

% first to invent, nyligen övergått till first inventor to file
När flera uppfinnare lämnar in ansökningar för samma uppfinning finns det olika strategier för att lösa tvisten. USA var under många år ett av få länder som applicerade \emph{first to invent} snarare än \emph{first to file}. 2013 bytte man till en variant av \emph{first to file} \cite{kravets}. Skillnaden mellan dessa system är vad som prioriteras när flera parter söker patent för samma sak: under \emph{first to invent} ska patentet gå till den som \emph{kom på} uppfinningen först oavsett när patentet ansöks; under \emph{first to file} ska patentet gå till den som först \emph{ansöker om patent} för uppfinningen oavsett när parten kom på den.

% förklara möjliga problem med FTI, FTF samt vad grace period är och har för poäng...
Ett problem med \emph{first to invent} vore om en uppfinnare väntade med att lämna in patentansökan tills någon annan gjorde detsamma, enbart för att behålla en särställning så länge som möjligt. Syftet med patent är ju att kunskapen ska överföras till allmänheten, så ett ytterligare krav är att uppfinnaren arbetat \emph{ihärdigt} med att beskriva sin uppfinning för att lämna in ansökan så snart som möjligt -- vilket blir ytterligare en bedömningsfråga vid en tvist.

%%%%%%%%%%%%%%%%%%%%%%%%%%%%%%%%%%%
%  Borde istället vara:
%  intro
%  hur funkat usa
%  hur funkat annars
%  problem usa: maska, skiljer sig från omvärlden
%  problem FTF: konkurrenter kan sabba (satsa resurser på söka snabbt 
%     alt public disclosure)
%  nu: "first-to-disclose", för att få folk att disclosa ivrigt.
%%%%%%%%%%%%%%%%%%%%%%%%%%%%%%%%%%%




% söktrycket har ökat markant
På senare år har antalet ansökningar ökat markant. Som en följd av detta har antalet patent som står i kö för granskning eskalerat kraftigt: från 2003 till 2006 har kön växt från 763~000 till 1~250~000, en ökning på över 60~\%.
