\section{Det moderna patentsystemet}
\label{sec:mod}

Här presenteras en kort redogörelse för viktiga händelser och avtal som ledde fram till det moderna patentsystem. Därefter sammanfattas det amerikanska, europeiska och kinesiska patentsystemen och slutligen berörs handeln med patent under rubriken \emph{Intelectual Property Rights}.

\subsection{Amerikanska patentsystemets utveckling på 1900-talet}

Med början runt 1930 blev domstolarna allt mer negativt inställda till
patent\cite{bracha}. Praxis att kräva ett \emph{flash of genius} blev vidspridd, i synnerhet efter fallet Cuno
Engineering v. Automatic Devices (1941). Domstolar ställde krav på att upptäckter genomförts i ett
genialt ögonblick, \emph{a flash of genius} -- således föll alla patent som bygde på succesiv förbättring och patentet måste tydligt uppvisa denna genialiska härkomst\cite{nard}. 

Patent Misuse Doctrine, som säger att om ett patent är för vitt till den grad att den hämmar legitim
konkurrens kan patentägaren ej stämma någon för intrång, användes allt oftare som försvar i rättsfall, i
synnerhet fallet Mercoid Corp. v. Mid-Continent Inv. Co.,\cite{nard}. Detta och annan praxis under
perioden gjorde det svårt att upprätthålla patent och precis som i England under sekelskiftet 1700,1800
blev det viktigt att se över lagstiftningen och konkretisera vad ett patent är och hur det ska behandlas.

%*Det är för oss inte klart vad att hämma legitim konkurrens innebär och 

Detta fortgick tills 1952 då patentsystemet konkretiserades avsevärt. Nu infördes
starkare krav på användbarhet, nyhet och icke-uppenbarhet. Det krävdes också en förklaring av vad
intrång till patentet innebär i ansökan. Till följd av detta tonades Patent Misuse Doctrine ned och
kravet på \emph{a flash of genius} upplöstes helt\cite{nard}.

Dock var patentmål fortfarande lösta i regionala domstolar och det fanns stora skillnader i hur olika
domstolar dömde i patentmål\cite{nard}. Det blev väldigt viktigt för båda pater att hålla målet i en
domstol som delade deras synpunkter på patentsystemet. Detta löstes först 1982 när the United States
Court of Appeals of the Federal Circuit skapades för att ta upp alla patentmål\cite{nard}. Instiftandet
av de nya domstolen gjorde patentlandskapet säkrare och stärkte patentens ställning. Studier visar att
efter instiftandet har patent ogiltigförklarats mindre ofta och fler sådana ogiltigförklaringar från
Regional Court har ogillats vid överlagan till FC. \cite{henry} Detta är relevant då patentmål i domstol
ofta handlar om varesig patentet ifråga, som målsägare anses har inträngts, är giltigt överhuvudtaget.
Om domstolen beslutar att patentet är gilltigt så behandlas eventuellt patentintrång. Då fler patent
fanns giltiga efter CAFC kunde fler dömas för patentintrång även om andelen som döms för intrång av
gilltigt patent ej ändrats särskilt mycket efter domstolens instiftning. 

%*Om detta seda när bra är en annan fråga. 
%*Vissa har dock höjt rösten ... nard

Stärkt validitet av patent är viktigt då risken att patentet avfärdas nu är lägre vid stämning för
intrång. Därför har patentägaren mindre att förlora i att stämma någon för intrång. Men
även om patentet erkänns som giltigt kan domstolen döma att den försvarande inte intränger.
Domstolen har då konkret dragit gränsen för patentets skydd, och detta kan i praktiken göra patentet
betydelselöst om domstolen finner att patentets skyddsområde är litet. Andra kan sedan gå i den frikända
inträngarens fotsteg utan efterdyningar\cite{henry}.

%*mer incitament till att ta patent och moderna begreppet patentportfolio. 



\subsection{Dagens patentsystem i USA}

I USA innebär ett patent att innehavaren kan \emph{exkludera} andra aktörer från marknaden. Mer specifikt kan denne förhindra andra från att \emph{tillverka}, \emph{använda} eller \emph{sälja} det som skyddats. Sedan internationella patentregler infördes 1996 gäller exklusiviteten även att \emph{erbjuda att sälja} samt att \emph{importera} \cite{cmu-overview}.


\subsubsection{Vem som kan söka patent}

Man skiljer på att komma på en idé och att omsätta idéer till praktik. Det är endast de som kommer på idéer som räknas som uppfinnare. Därtill måste man vara själva uppfinnaren för att få patent; ett företag som anställer någon kan inte söka dennes patent, inte heller någon som finansierar forskningen. Uppfinnaren signerar patentet tillsammans med ett löfte på heder och samvete att man är den sanne uppfinnaren \cite{cmu-overview}.

Alla som står med på patentansökan måste vara verkliga uppfinnare, och därtill måste alla verkliga uppfinnare stå med på patentansökan. Vem som helst, oavsett nationalitet, får söka.

%ove säger att amn kan sälja och överföra patent mellan personer och företag (en uppfinnare kn ge patent till sitt företag)


\subsubsection{Vad man kan söka patent för}

Man skiljer på tre olika typer av patent. Den vanligaste och mest kända typen av patent är \emph{utility patent} \cite{cmu-overview}.
% * Referera här istället till listan av de generella kraven på patent.
%% Kraven är att det som patenteras är \emph{nytt}, \emph{användbart} och \emph{ej är uppenbart}. 
Därtill finns \emph{växtpatent} på växter som man skapat asexuellt samt designpatent som rör utseende men ej funktion eller struktur. Designpatent är i praktiken mycket smala, och förhindrar bara rena kopior \todo{Har slarvat bort min källa där det stod en anekdot om Google}. Den maximala skyddstiden för designpatent är 14 år; skyddstiden för de övriga typerna av patent ökade år 1996 från 17 till 20 år.


\subsubsection{Från first to invent mot first to file} 

När flera uppfinnare lämnar in ansökningar för samma uppfinning finns det olika strategier för att lösa tvisten. USA var under många år ett av få länder som applicerade \emph{first to invent} snarare än \emph{first to file}. 2013 bytte man till en variant av \emph{first to file} \cite{leahy}.%byta till riktig källa
% går att byta till en riktig källa. rätt säker på att allt förklaras av kravets, men antar att du inte är ett fan av att ta fakta från en så "subjektiv" artikel.
Skillnaden mellan dessa system är vad som prioriteras när flera parter söker patent för samma sak: under \emph{first to invent} ska patentet gå till den som \emph{kom på} uppfinningen först oavsett när patentet ansöks. Under \emph{first to file} ska patentet istället gå till den som först \emph{ansöker om patent} för uppfinningen oavsett när vederbörande kom på den. 

Sedan mars 2013 används ett nytt system som professor L. Kravets i en nyhetsartikel kallar \emph{first to disclose} \cite{kravets}. Systemet grundar sig på \emph{first to file}, men har några viktiga skillnader. Även om en uppfinnare delger sina idéer till allmänheten kan patent sökas inom ett år. Dessutom prioriteras den \emph{först} delgav sina idéer till allmänheten om flera parter ansöker om patent för samma upptäckt (därav first to \emph{disclose}). Detta ger incitament att sprida sina idéer så fort som möjligt, vilket står i tydlig kontrast med \emph{first to invent}-systemet.


%\subsubsection{Amerikanska patentverket dignar under belastningen}
%
%På senare år har antalet ansökningar ökat markant. Som en följd av detta har antalet patent som står i kö för granskning eskalerat kraftigt: från 2003 till 2011 har kön växt från 763~000 till 1~250~000, en ökning på över 60~\%.

%källor:
% http://www.uspto.gov/about/stratplan/ar/2003/060405_table5.jsp
% http://www.uspto.gov/about/stratplan/ar/2011/oai_05_wlt_05.html