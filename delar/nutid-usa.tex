\subsection{Dagens patentsystem i USA}

I USA innebär ett patent att innehavaren kan \emph{exkludera} andra aktörer från marknaden. Mer specifikt kan denne förhindra andra från att \emph{tillverka}, \emph{använda} eller \emph{sälja} det som skyddats. Sedan internationella patentregler infördes 1996 gäller exklusiviteten även att \emph{erbjuda att sälja} samt att \emph{importera} \cite{cmu-overview}.

\subsubsection{Vem som kan söka patent}
Man skiljer på att komma på en idé och att omsätta idéer till praktik. Det är endast de som kommer på idéer som räknas som uppfinnare. Därtill måste man vara själva uppfinnaren för att få patent; ett företag som anställer någon kan inte söka dennes patent, inte heller någon som finansierar forskningen. Uppfinnaren signerar patentet tillsammans med ett löfte på heder och samvete om att man är den sanne uppfinnaren \cite{cmu-overview}.

Alla som står med på patentansökan måste vara verkliga uppfinnare, och därtill måste alla verkliga uppfinnare stå med på patentansökan. Vem som helst, oavsett nationalitet, får söka.

\subsubsection{Vad man kan söka patent för}
Man skiljer på tre olika typer av patent. Den vanligaste och mest kända typen av patent är \emph{utility patent} \cite{cmu-overview}.
\todo{Referera här istället till listan av de generella kraven på patent.}
% Kraven är att det som patenteras är \emph{nytt}, \emph{användbart} och \emph{ej är uppenbart}. 
Därtill finns \emph{växtpatent} på växter som man skapat asexuellt samt designpatent som rör utseende men ej funktion eller struktur. Designpatent är i praktiken mycket smala, och förhindrar bara rena kopior \todo{Har slarvat bort min källa där det stod en anekdot om Google}. Den maximala skyddstiden för designpatent är 14 år; skyddstiden för de övriga typerna av patent ökade år 1996 från 17 till 20 år.

\subsubsection{Från first to invent mot first to file}
När flera uppfinnare lämnar in ansökningar för samma uppfinning finns det olika strategier för att lösa tvisten. USA var under många år ett av få länder som applicerade \emph{first to invent} snarare än \emph{first to file}. 2013 bytte man till en variant av \emph{first to file} \cite{kravets}. Skillnaden mellan dessa system är vad som prioriteras när flera parter söker patent för samma sak: under \emph{first to invent} ska patentet gå till den som \emph{kom på} uppfinningen först oavsett när patentet ansöks; under \emph{first to file} ska patentet gå till den som först \emph{ansöker om patent} för uppfinningen oavsett när parten kom på den.

Ett problem med \emph{first to invent} vore om en uppfinnare väntade med att lämna in patentansökan tills någon annan gjorde detsamma, enbart för att behålla en särställning så länge som möjligt. Syftet med patent är ju att kunskapen ska överföras till allmänheten, så ett ytterligare krav är att uppfinnaren arbetat \emph{ihärdigt} (eng. \emph{diligently}) med att beskriva sin uppfinning för att lämna in ansökan så snart som möjligt -- vilket blir ytterligare en bedömningsfråga vid en tvist. Trots detta krav har uppfinnaren incitament att dra ut på processen att skriva patentansökan så mycket som möjligt. Detta, tillsammans med det faktum att systemet är tvärt emot hur det fungerar i resten av världen, sporrade förändring.

Tanken med \emph{first to file} är att skapa incitament för att skriva sin ansökan så snabbt som möjligt. Det finns dock problem även här. Konkurrenter som får nys om en uppfinning kan sabotera för verkliga uppfinnarna genom att satsa resurser på att ansöka om patent ännu snabbare, alternativt att delge idéerna till allmänheten för att på så sätt förhindra att konkurrenten får ensamrätt.

Sedan mars 2013 används ett nytt system som L. Kravets kallar \emph{first to disclose} \cite{kravets}. Systemet grundar sig på \emph{first to file}, men har några viktiga skillnader. Även om en uppfinnare delger sina idéer till allmänheten kan patent sökas inom ett år. Inte nog med det -- om flera uppfinnare söker patent för samma sak är det den som \emph{först} delgav sina idéer som får patentet (därav \emph{``first to disclose''}). Detta ger incitament att sprida sina idéer så fort det bara går, vilket är poängen med patent och står i tydlig kontrast med \emph{first to invent}-systemet.


\subsubsection{Amerikanska patentverket dignar under belastningen}

På senare år har antalet ansökningar ökat markant. Som en följd av detta har antalet patent som står i kö för granskning eskalerat kraftigt: från 2003 till 2011 har kön växt från 763~000 till 1~250~000, en ökning på över 60~\%.
