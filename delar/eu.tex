\subsection{Patent i Europa}

Det finns i dagsläget två vägar för att söka patent i Europa:
Antingen söker man nationella patent i de länder där man önskar skydd, eller så söker man patent vid det europeiska patentverket, \emph{the European Patent Office}, EPO.
En ansökan till EPO granskas och prövas av EPO, som sen kan bevilja patent i medlemsländerna.
Man gör alltså bara en ansökan, men om patentet beviljas måste översättningar av patentet lämnas in, och avgifter erläggas, i de länder där patentet ska gälla.
EPO är oberoende EU, men alla EU:s medlemsnationer och flera andra europeiska länder är anslutna.

Båda dessa sätt att ta patent är dock mycket dyrare än i resten av världen.
En uppskattning är att det kostar ca tjugo gånger mer att skydda en uppfinning i EU än i USA.
Oavsett vilken väg som väljs multipliceras avgifter och översättningskostnader för nära trettio länder och gör det minst sagt kostsamt
att patentera en uppfinning i Europa.
För att begränsa kostnaderna är det vanligt att endast ansöka om patent i ett strategiskt urval av länder.
På så vis uppnås ett acceptabelt skydd mot en märkbart lägre kostnad.

År 2012 gjordes dock ett genombrott för patent i Europa.
Vid Europeiska rådet nåddes en slutgiltig överenskommelse om det så kallade enhetliga patentskyddet. 
Överenskommelsen omfattar 25 länder, alla EU:s medlemmar vid tidpunkten utom Spanien och Italien. 
Patent med enhetligt skydd, \emph{unitary effect}, ansöks om vid EPO på något av tre officiella språk.
Därefter behöver inga fler översättningar göras för hand, utan man låter istället översätta patentet maskinellt för att användas som referens vid patentsökningar.
Man erlägger inte heller några avgifter till de nationella patentverken.
Fördelarna med så kallade \emph{unitary patent} är ett förenklat förfarande vid ansökan, minskade kostnader, och ett enhetlig juridisk effekt.
Även efter dessa förändringar så blir det flera gånger dyrare att söka patent i EU än i exempelvis USA och Kina.

Den europeiska patentkonventionen, EPC, överensstämmer med GATT i de flesta avseenden.
En den utmärkande skillnad är att EPC artikel 52 fastslår att datorprogram inte är uppfinningar, och därför inte är patenterbara.
Samma artikel utesluter även patent på växter och djur.
Liksom de flesta andra länders patentsystem kan EPC sägas utöva \emph{first to file}-principen, då nyheten av ett patent bedöms även i ljuset av tidigare inlämnade, men ännu ej beviljade patentansökningar.


\begin{itemize}
	%\item Patent skall vara tillgängliga för uppfinningar (produket eller process) inom alla områden så länge ett "inventive step" har tagits och uppfinningen är applicerbar.
	%\item Patentägaren har rätt att stoppa tredje part från att producera, sälja, importera skyddad produkt eller motsvarande för en process.
	%\item Patenttagaren måste publicera uppfinngen tydligt.
	\item Patentet varar i 20 år.
	\item Växter och djur är ej patenterbara.
	\item Datorprogram är ej patenterbara.
\end{itemize}	


% EPC:
% documents.epo.org/projects/babylon/eponet.nsf/0/00E0CD7FD461C0D5C1257C060050C376/$File/EPC_15th_edition_2013.pdf

% ec.europa.eu/internal_market/indprop/patent/faqs/index_en.htm#maincontentSec1

% EPO-patent:
% http://www.epo.org/applying/basics.html

% Unitary Patent:
% http://www.epo.org/law-practice/unitary/unitary-patent.html

% Bleh: search.proquest.com.proxy.lib.chalmers.se/docview/1239076071