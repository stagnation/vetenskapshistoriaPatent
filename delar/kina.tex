
\subsection{Kina}

Immaterialrättslagstiftning är förhållandevis nytt i Kina. Första patentlagen stiftades 1984, och första upphovsrättslagen kom 1990 \cite{gergils}. Patentlagen har justerats några gånger, inklusive en revision inför att landet gick med i WTO år 2001. Detta innebär att själva lagstiftningen är sofistikerad, men som Håkan Gergils -- Senior Advisor på Kungl. Ingenjörsvetenskapsakademien (IVA) -- kommenterar: ``Trots att Kina på pappret har en patenträtt som uppfyller västerländska mått, så är 
efterlevnaden inte alls lika god''.

I verkligheten träder en bild fram där man skyddar sig inte främst genom att söka patent och ställa konkurrenter inför rätta, utan där alternativa grepp tas. Ordish skriver ``it is no secret that IP protection in China is a challenge'' \cite{ordish} i inledningen till en artikel om hur man skyddar sina immateriella rättigheter. Han beskriver flera metoder, bland annat kan ett företag undersöka den kinesiska parten noga samt bygga en stark relation för att de inte ska vilja utnyttja företagets immateriella tillgångar.

Vidare undersöker tre forskare i Schweiz hur IPR upprätthålls i praktiken genom att utföra djupgående intervjuer av företagsledare i olika brancher i Kina. Olika de facto-strategier framträder \cite{keupp}; huvudsakligen antingen genom att göra produkten tekniskt svårkopierad, hålla kunskapen odokumenterad och hos få personer, utnyttja sociala nät inom företaget och till korrupta tjänstemän, eller helt enkelt genom att se otillåtna kopior som gratisreklam och trycka på att det finns en stor vinst i kvalitet hos originalet.

Trots detta ökar antalet sökta patent i Kina i hög takt precis som övriga stora länder (se figur \ref{fig:holg}), och utländska företag investerar mycket i landet -- även inom R\&D (UNCTAD 2005 via Keupp et al). Guangzhou Hu rapporterar att en stor del av de kinesiska patenten kommer från utländska företag och att dessa i stor utsträckning tar patent som en del av konkurrensen sinsemellan\cite{hu}. Rapporten finner också att ökad konkurrerande import till Kina leder till att utländska företag söker fler patent.

Denna ökning av utländska patent kommer efter att Kina implementerat ett rigidt patentsystem och parallellt med att produktion i Kina och den kinesiska marknaden blir allt viktigare för internationella företag. Guangzhou Hu finner också att 90\% av de utlänska patentent hävdar \emph{foreign priority} -- vilket indikerar att företaget redan har ansökt om patentet i utlandet\cite{hu}. 
