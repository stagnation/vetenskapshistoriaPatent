
\subsection{Kina}

Immaterialrättslagstiftning är förhållandevis nytt i Kina. Första patentlagen stiftades 1984, och första upphovsrättslagen kom 1990 \cite{gergils}. Patentlagen har justerats några gånger, inklusive en revision inför att landet gick med i WTO år 2001. Detta innebär att själva lagstiftningen är sofistikerad, men som Håkan Gergils -- Senior Advisor på IVA -- kommenterar: ``Trots att Kina på pappret har en patenträtt som uppfyller västerländska mått, så är 
efterlevnaden inte alls lika god''.

I verkligheten träder en bild fram där man skyddar sig inte främst genom att söka patent och dra konkurrenter inför rätta, utan där alternativa grepp tas. Ordish skriver ``it is no secret that IP protection in China is a challenge'' \cite{ordish} i inledningen till en artikel om hur man skyddar sina immateriella rättigheter, bland annat genom att undersöka den kinesiska parten noga samt bygga en stark relation för att de inte ska vilja utnyttja ens immateriella tillgångar.

Vidare undersöker tre forskare i Schweiz hur IPR upprätthålls i praktiken genom att utföra djupgående intervjuer av företagsledare i olika brancher i Kina. Olika de facto-strategier framträder \cite{keupp}; huvudsakligen antingen genom att göra produkten tekniskt svårkopierad, hålla kunskapen odokumenterad och hos få personer, utnyttja sociala nät inom företaget och till korrupta tjänstemän, eller helt enkelt genom att se otillåtna kopior som gratis reklam och trycka på att det finns en stor vinst i kvalitet hos originalet.

Trots detta ökar antalet sökta patent i Kina i hög takt precis som övriga stora länder (se figur \ref{fig:holg}), och utländska företag investerar mycket i landet -- även inom R\&D (UNCTAD 2005 via Keupp et al).
