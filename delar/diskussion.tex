\section{Diskussion och avslutning}
\label{sec:disk}



Vi kan nu besvara rapportens frågeställningar om hur patent skiftade från ett kungligt privilegium till uppfinnarens rättighet att skydda sin upptäckt. Vi diskuterar vilka som har haft makt över patentsystemet genom historien och berör skillnaderna i vem som gynnas av \emph{first to file} och \emph{first to invent}. Slutligen utvärderar vi om det moderna patentsystemet kan beskrivas med maximet att ett patent är ett kontrakt mellan en
uppfinnare och allmänheten där allmänheten tillåter att uppfinnaren har ensam
ekonomisk vinning av uppfinningen i utbyte mot att han delger kunskap till
allmänheten.


\subsection{Förskjutning från ett kungligt privilegium till uppfinnarens rättighet}
Vi har sett att patent har skiftat helt ifrån ett privilegium som beviljades av kungen till en formell rättighet som skyddar uppfinnares idéer. I medeltidens England användes de kungliga privilegierna för att styra marknaden och ge incitament eller belöning till de entreprenörer som kommersialiserade nya marknader importerade tekniska lösningar till den engelska marknaden. Det går inte att dra ett klart streck i vem som gynnades av dessa privilegier. I många fall gynnades folket, se framförallt tygindustrin på 1300-talet. Men i andra fall var privilegierna enbart utgivna för att gynna kungen eller hans uppbackare.%*ordval.
Då vi har inte funnit något patent från den här tiden som missgynnade kungen, sammanfattar vi denna epok som en tid då kungen gynnades av patentsystemet. Detta missbruk av kungliga privilegier mynnade ut i införandet av Statue of Monopolies 1624, men att uppfinningspatent exkluderades ur det juridiska förbudet på privilegier indikerar att folket inte missgynnades allt för mycket under privilegiesystemet för uppfinningar.

Privilegiesystemet för att bevilja uppfinningspatent levde kvar ända tills mitten av 1800-talet i england och USA. Under dessa tre hundra år hade patent gradvis skiftat från att beröra entreprenörskap till att beröra uppfinnande och parallellt skiftade även perspektivet på patent som ett utgivet privilegium till en uppfinnarens rättighet att ha ensamrätt till sin upptäckt. Samtidigt hade den politiska dragkampen mellan kungen och regeringen i england gjort att kungen förlorat än mer makt i att ge ut patent för uppfinningar. Vilket ledde till att färre och färre patent avslogs vilket urvattnade värdet av privilegiet som ett patent innebar, om alla som ansökte om patent beviljades så var det inte så viktigt. Dessa är två viktiga anledningar till att patent kom att betraktas som en rättighet i england.

I USA å andra sidan hade kolonierna kopierat englands privilegiesystem och först vid staternas enande började utvecklingen mot ett rättighetsperspektiv. Här saknades den engelska motsättningen mellan kung som gav ut patent och regering som kunde reglera patent i lag och också sökte skära ner på kungens makt. I USA höll staten båda rollerna; att lagstifta om patent och att bevilja privilegiepatent så skiftet mot ett rättighetsperspektiv genomfördes efter att 1793 års lagstifning möjliggjort för fyra årtionden av godvilliga patent. Allmänheten insåg behovet av att göra om systemet då privilegiemodellen ej fungerade och då USA sedan konstitutionen haft ett starkare uppfinningsperspektiv (de hade \emph{first to invent} som prioriteringsordning) kom rättighetsperspektivet att lysa igenom.


\subsection{Kinas framtida patentskydd}

Kina är ett intressant fall, å ena sidan så ökar patentansökningarna lavinartat, till stor del från internationella företag som importerar befintliga patent till Kina. Men då patentskydd historiskt sett saknats finns det  utbredd praxis för hur patentsystemet skall kringgås. Vi har inte hittat någon källa som jämför dessa två motsatta system, där internationella företag använder det internationellt vedertagna juridiska patentsystemet och kinesiska företag kringgår detta på ett metodiskt sätt. 

Det ska bli intressant att se vad som händer i framtiden om tillströmmen av utländska patent kommer sakta ned eller skifta från import av tidigare patent mot helt nya uppfinningar. Vi undrar också om den klassiska kinesiska kopieringskulturen kommer tonas ned om Kina ytterligare stärker upprätthållandet av patent och \emph{IPR}.

\subsection{Patent som ekonomiskt styrmedel}
Patent uppkom i stort med syftet att styra marknaden, med tiden utvecklades ett mer demokratiskt system som skyddar uppfinnarens idéer och sprider kunskap till allmänheten. Men modern utveckling  av patent och \emph{IPR} har lett till att patent återigen utnyttjas som ett styrmedel för marknaden. Nu är det inte en kung som styr marknadens aktörer genom privilegier och restriktioner för att gynna sitt land och sitt folk. Utan det är marknadsaktörerna själva som använder patent för att dra gränser i hur konkurrenter kan utvecklas och bedriva verksamhet. Det bedrivs också mycket lobbying för att definiera patent så gynnsamt om möjligt för storföretagen, se Pfizers inblandning i utvecklandet av \emph{GATT}.

En viktig skillnad mellan hur privilegiesystemet och modern patenträtt utnyttjas för att styra marknaden är att på medeltiden erbjöds privilegium för att locka till sig kompetens från utlandet för att den inhemska marknaden skulle kunna konkurrera och i kombination med skyddstullar kunde utländska konkurrenter helt uteslutas från marknaden. Detta kan ses som att man stal kunskap från andra företag och gav den till ett fåtal lärlingar och inhemska arbetare. Nu används patent för att begränsa i vilka riktningar konkurrenter kan utveckla sina produkter och för att öppna forskningsvägar för det egna företaget. I och med den moderna specifikationen så delges nu kunskap till hela allmänheten. 

Om det nutida systemet är gynnsamt för marknaden eller ej är svårt att säga, men vi kan konstatera att mycket större kunskap sprids till allmänheten i det moderna patentsystemet än medeltidens privilegiesystem. Men att ansöka om ett patent är fortfarande dyrt och ansträngande varför det är svårt för en enskild individ att söka och upprätthålla patent. Varför vi anser att patentsystemet främst gynnar företag som har råd att använda patent som ett strategiskt styrmedel dock är det nu mycket bättre för allmänheten än privilegiesystemet men att det finns möjligheter till förbättring.

\subsection{Skillnader mellan \emph{first to invent} och \emph{first to file}}

USA har nyligen bytt från \emph{first to invent} till en variant av \emph{first to file}. Ett problem med \emph{first to invent} vore om en uppfinnare väntade med att lämna in patentansökan tills någon annan gjorde detsamma, enbart för att behålla en särställning så länge som möjligt. Därav är ett ytterligare krav att uppfinnaren arbetat \emph{ihärdigt} (eng. \emph{diligently}) med att beskriva sin uppfinning för att lämna in ansökan så snart som möjligt\cite{cmu-overview} -- vilket blir ytterligare en bedömningsfråga vid en tvist\cite{cmu-overview}. Trots detta krav har uppfinnaren egenintresse i att dra ut ansökansprocessen så mycket som möjligt.

Denna modell är djupt rotad i perspektivet att patent är en rättighet som ges till en uppfinnare, men det överenstämmer ej med perspektivet att patent är ett kontrakt mellan uppfinnaren och allmänheten. Under \emph{first to invent} är det snarare den som uppfinner någonting näst först, ofta ovetande om att det redan är uppfunnet, som direkt och indirekt delger informationen till allmänheten. Dels genom sin egen patentansökan men framförallt den ansökan som den första uppfinnaren lämnar in i respons. Och det är den första uppfinnaren som får belöningen av ett patent, enbart för att vederbörande gjorde en upptäckt, ej att vederbörande delgav den till allmänheten.

Tanken med \emph{first to file} är att skapa incitament för att skriva sin ansökan så snabbt som möjligt. Det finns dock problem även här. Konkurrenter som får nys om en uppfinning kan sabotera för verkliga uppfinnarna genom att satsa resurser på att lämna in en ansökan först, alternativt att delge idéerna till allmänheten för att på så sätt förhindra att någon kan söka patent. Då det ej går att patentera något som redan är känt, med en viss tidsfrist.

Denna model ligger närmre till synsättet att allmänheten erbjuder ett patent i utbyte mot att uppfinnare delger sina upptäckter. Men den amerikanska varianten som kan kallas \emph{first to disclose} lägger ytterligare vikt på att ge ut informationen till allmänheten. Då patentprioritering baseras på första datumet uppfinnaren delgav informationen till allmänheten, innan denne senare ansökte om ett patent.


\subsection{Det moderna patentbegreppet}

Rapporten inleddes med en definition av patentbegreppet med grund i hur det betraktades under medeltiden. Nu har vi kommit till den moderna definitionen där patent är ett skydd som beviljas uppfinnaren till en teknisk produkt eller lösning som uppfyller ett antal krav på nyhet och icke-uppenbarhet. Detta patent tillåter patentägaren att utesluta andra aktörer från att bland annat sälja och producera produkten eller i fallet en process, omsätta den för ekonomisk vinning. Vidare kan patent sälja, licensieras och överlåtas för att aktörer skall kunna samverka. Patent är en del i ett bredare området immaterialrätt.

Den moderna lagstiftningen är tydlig i att uppfinnaren skall delge kunskap till allmänheten och att uppfinningen juridiskt skyddas. Efter successiva prissänkningar av patentkostnaden och stärkning av patent är patent nu inom räckhåll för uppfinnare, men det finns fortfarande mycket juridiska hinder och strategier som större företag kan uttnytja för att hindra den enskilde patenttagarens kommersialisering. Men i och med att han kan sälja sitt patent till ett företag konstaterar vi att detta maxim, ett patent är ett kontrakt mellan en
uppfinnare och allmänheten där allmänheten tillåter att uppfinnaren har ensam
ekonomisk vinning av uppfinningen i utbyte mot att han delger kunskap till
allmänheten, är en lämplig beskrivning idag.